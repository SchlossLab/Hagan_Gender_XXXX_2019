\documentclass[11pt,]{article}
\usepackage{lmodern}
\usepackage{amssymb,amsmath}
\usepackage{ifxetex,ifluatex}
\usepackage{fixltx2e} % provides \textsubscript
\ifnum 0\ifxetex 1\fi\ifluatex 1\fi=0 % if pdftex
  \usepackage[T1]{fontenc}
  \usepackage[utf8]{inputenc}
\else % if luatex or xelatex
  \ifxetex
    \usepackage{mathspec}
  \else
    \usepackage{fontspec}
  \fi
  \defaultfontfeatures{Ligatures=TeX,Scale=MatchLowercase}
\fi
% use upquote if available, for straight quotes in verbatim environments
\IfFileExists{upquote.sty}{\usepackage{upquote}}{}
% use microtype if available
\IfFileExists{microtype.sty}{%
\usepackage{microtype}
\UseMicrotypeSet[protrusion]{basicmath} % disable protrusion for tt fonts
}{}
\usepackage[margin=1.0in]{geometry}
\usepackage{hyperref}
\hypersetup{unicode=true,
            pdftitle={Supplementary},
            pdfborder={0 0 0},
            breaklinks=true}
\urlstyle{same}  % don't use monospace font for urls
\usepackage{graphicx,grffile}
\makeatletter
\def\maxwidth{\ifdim\Gin@nat@width>\linewidth\linewidth\else\Gin@nat@width\fi}
\def\maxheight{\ifdim\Gin@nat@height>\textheight\textheight\else\Gin@nat@height\fi}
\makeatother
% Scale images if necessary, so that they will not overflow the page
% margins by default, and it is still possible to overwrite the defaults
% using explicit options in \includegraphics[width, height, ...]{}
\setkeys{Gin}{width=\maxwidth,height=\maxheight,keepaspectratio}
\IfFileExists{parskip.sty}{%
\usepackage{parskip}
}{% else
\setlength{\parindent}{0pt}
\setlength{\parskip}{6pt plus 2pt minus 1pt}
}
\setlength{\emergencystretch}{3em}  % prevent overfull lines
\providecommand{\tightlist}{%
  \setlength{\itemsep}{0pt}\setlength{\parskip}{0pt}}
\setcounter{secnumdepth}{0}
% Redefines (sub)paragraphs to behave more like sections
\ifx\paragraph\undefined\else
\let\oldparagraph\paragraph
\renewcommand{\paragraph}[1]{\oldparagraph{#1}\mbox{}}
\fi
\ifx\subparagraph\undefined\else
\let\oldsubparagraph\subparagraph
\renewcommand{\subparagraph}[1]{\oldsubparagraph{#1}\mbox{}}
\fi

%%% Use protect on footnotes to avoid problems with footnotes in titles
\let\rmarkdownfootnote\footnote%
\def\footnote{\protect\rmarkdownfootnote}

%%% Change title format to be more compact
\usepackage{titling}

% Create subtitle command for use in maketitle
\newcommand{\subtitle}[1]{
  \posttitle{
    \begin{center}\large#1\end{center}
    }
}

\setlength{\droptitle}{-2em}

  \title{Supplementary}
    \pretitle{\vspace{\droptitle}\centering\huge}
  \posttitle{\par}
    \author{}
    \preauthor{}\postauthor{}
    \date{}
    \predate{}\postdate{}
  
\usepackage{booktabs}
\usepackage{longtable}
\usepackage{array}
\usepackage{multirow}
\usepackage[table]{xcolor}
\usepackage{wrapfig}
\usepackage{float}
\usepackage{colortbl}
\usepackage{pdflscape}
\usepackage{tabu}
\usepackage{threeparttable}
\usepackage{threeparttablex}
\usepackage[normalem]{ulem}
\usepackage{makecell}
\usepackage{caption}

\usepackage{helvet} % Helvetica font
\renewcommand*\familydefault{\sfdefault} % Use the sans serif version of the font
\usepackage[T1]{fontenc}

\usepackage[none]{hyphenat}

\usepackage{setspace}
\doublespacing
\setlength{\parskip}{1em}

\usepackage{lineno}

\usepackage{pdfpages}
\floatplacement{figure}{H} % Keep the figure up top of the page
\usepackage{booktabs}
\usepackage{longtable}
\usepackage{array}
\usepackage{multirow}
\usepackage{wrapfig}
\usepackage{float}
\usepackage{colortbl}
\usepackage{pdflscape}
\usepackage{tabu}
\usepackage{threeparttable}
\usepackage{threeparttablex}
\usepackage[normalem]{ulem}
\usepackage{makecell}
\usepackage{xcolor}

\begin{document}
\maketitle

\section{Gender representation}\label{gender-representation}

\includegraphics{supp_inst_stats.png}

Supplemental Figure . Break down of individuals in each role by their
gender and US institute type. Authors are restricted to senior authors
(e.g., corresponding, last).

\newpage

\section{Gender bias}\label{gender-bias}

\includegraphics{Supp_inst.png}

Supplemental Figure . Difference in acceptance and rejections by
institution type.

\newpage

\includegraphics{Supp_time.png}

Supplemental Figure . Total days and versions between submission and
final rejection at each journal.

\newpage

\subsection{Gender prediction and
assignment}\label{gender-prediction-and-assignment}

\includegraphics{genderize_method.png}

Supplemental Figure . Schematic of gender prediction and assignment.

\newpage

\subsection{Validating gender
analysis}\label{validating-gender-analysis}

Supplemental Table 1. sensitivity/specificity/accuracy of genderize
thresholds. Bolded text denotes the accuracy of the threshold used in
all further analyses.

\begin{table}[H]
\centering
\begin{tabular}{l|r|r|l|r|r|l}
\hline
\multicolumn{1}{c|}{ } & \multicolumn{3}{c|}{First Names} & \multicolumn{3}{c}{Plus Country Data} \\
\cline{2-4} \cline{5-7}
Measure & p0.5 & p0.85 & pmod0.85 & p0.5 & p0.85 & pmod0.85\\
\hline
Sensitivity & 0.8943 & 0.9516 & \cellcolor{white}{0.971} & 0.9055 & 0.9471 & \cellcolor{white}{0.9669}\\
\hline
Specificity & 0.9339 & 0.9593 & \cellcolor{white}{0.972} & 0.9265 & 0.9553 & \cellcolor{white}{0.9727}\\
\hline
Accuracy & 0.9110 & 0.9549 & \textbf{0.9714} & 0.9146 & 0.9507 & \textbf{0.9695}\\
\hline
\end{tabular}
\end{table}

\vspace{40mm}

\includegraphics{impact_equation.png}

Supplemental Figure . Equation for calculating negative bias by
genderize. C indicates an individual country.

\newpage

\includegraphics{Supp_genderize_3.png}

Supplemental Figure . The negative impact of each country on the overall
gender prediction of the validation dataset. Number indicates the total
number of names associated with each country.

\newpage

\includegraphics{Supp_genderize_4.png}

Supplemental Figure . The negative impact of each country on the overall
gender prediction of the full dataset. Number indicates the total number
of names associated with each country.


\end{document}
