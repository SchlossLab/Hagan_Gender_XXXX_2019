\documentclass[11pt,]{article}
\usepackage{lmodern}
\usepackage{amssymb,amsmath}
\usepackage{ifxetex,ifluatex}
\usepackage{fixltx2e} % provides \textsubscript
\ifnum 0\ifxetex 1\fi\ifluatex 1\fi=0 % if pdftex
  \usepackage[T1]{fontenc}
  \usepackage[utf8]{inputenc}
\else % if luatex or xelatex
  \ifxetex
    \usepackage{mathspec}
  \else
    \usepackage{fontspec}
  \fi
  \defaultfontfeatures{Ligatures=TeX,Scale=MatchLowercase}
\fi
% use upquote if available, for straight quotes in verbatim environments
\IfFileExists{upquote.sty}{\usepackage{upquote}}{}
% use microtype if available
\IfFileExists{microtype.sty}{%
\usepackage{microtype}
\UseMicrotypeSet[protrusion]{basicmath} % disable protrusion for tt fonts
}{}
\usepackage[margin=1.0in]{geometry}
\usepackage{hyperref}
\hypersetup{unicode=true,
            pdftitle={Women are underrepresented and receive differential outcomes at ASM journals: A six-year retrospective analysis},
            pdfborder={0 0 0},
            breaklinks=true}
\urlstyle{same}  % don't use monospace font for urls
\usepackage{longtable,booktabs}
\usepackage{graphicx}
% grffile has become a legacy package: https://ctan.org/pkg/grffile
\IfFileExists{grffile.sty}{%
\usepackage{grffile}
}{}
\makeatletter
\def\maxwidth{\ifdim\Gin@nat@width>\linewidth\linewidth\else\Gin@nat@width\fi}
\def\maxheight{\ifdim\Gin@nat@height>\textheight\textheight\else\Gin@nat@height\fi}
\makeatother
% Scale images if necessary, so that they will not overflow the page
% margins by default, and it is still possible to overwrite the defaults
% using explicit options in \includegraphics[width, height, ...]{}
\setkeys{Gin}{width=\maxwidth,height=\maxheight,keepaspectratio}
\IfFileExists{parskip.sty}{%
\usepackage{parskip}
}{% else
\setlength{\parindent}{0pt}
\setlength{\parskip}{6pt plus 2pt minus 1pt}
}
\setlength{\emergencystretch}{3em}  % prevent overfull lines
\providecommand{\tightlist}{%
  \setlength{\itemsep}{0pt}\setlength{\parskip}{0pt}}
\setcounter{secnumdepth}{0}
% Redefines (sub)paragraphs to behave more like sections
\ifx\paragraph\undefined\else
\let\oldparagraph\paragraph
\renewcommand{\paragraph}[1]{\oldparagraph{#1}\mbox{}}
\fi
\ifx\subparagraph\undefined\else
\let\oldsubparagraph\subparagraph
\renewcommand{\subparagraph}[1]{\oldsubparagraph{#1}\mbox{}}
\fi

%%% Use protect on footnotes to avoid problems with footnotes in titles
\let\rmarkdownfootnote\footnote%
\def\footnote{\protect\rmarkdownfootnote}

%%% Change title format to be more compact
\usepackage{titling}

% Create subtitle command for use in maketitle
\providecommand{\subtitle}[1]{
  \posttitle{
    \begin{center}\large#1\end{center}
    }
}

\setlength{\droptitle}{-2em}

  \title{\textbf{Women are underrepresented and receive differential outcomes at
ASM journals: A six-year retrospective analysis}}
    \pretitle{\vspace{\droptitle}\centering\huge}
  \posttitle{\par}
    \author{}
    \preauthor{}\postauthor{}
    \date{}
    \predate{}\postdate{}
  
\usepackage{booktabs}
\usepackage{longtable}
\usepackage{array}
\usepackage{multirow}
\usepackage[table]{xcolor}
\usepackage{wrapfig}
\usepackage{float}
\usepackage{colortbl}
\usepackage{pdflscape}
\usepackage{tabu}
\usepackage{threeparttable}
\usepackage{threeparttablex}
\usepackage[normalem]{ulem}
\usepackage{makecell}
\usepackage{caption}

\usepackage{helvet} % Helvetica font
\renewcommand*\familydefault{\sfdefault} % Use the sans serif version of the font
\usepackage[T1]{fontenc}

\usepackage[none]{hyphenat}

\usepackage{setspace}
\doublespacing
\setlength{\parskip}{1em}

\usepackage{lineno}
\usepackage{iftex}

\usepackage{pdfpages}
\floatplacement{figure}{H} % Keep the figure up top of the page
\usepackage{booktabs}
\usepackage{longtable}
\usepackage{array}
\usepackage{multirow}
\usepackage{wrapfig}
\usepackage{float}
\usepackage{colortbl}
\usepackage{pdflscape}
\usepackage{tabu}
\usepackage{threeparttable}
\usepackage{threeparttablex}
\usepackage[normalem]{ulem}
\usepackage{makecell}
\usepackage{xcolor}

\begin{document}
\maketitle

\vspace{35mm}

Running title: A six-year retrospective analysis of ASM journal outcomes

\vspace{35mm}

Ada K. Hagan\(^{*1\dagger}\), Begüm D. Topçuoğlu\({^1}\), Mia E.
Gregory\({^1}\), Hazel A. Barton\({^2}\), Patrick D.
Schloss\textsuperscript{1\(\dagger\)}

\vspace{30mm}

\(\dagger\) To whom correspondence should be addressed:
\href{mailto:pschloss@umich.edu}{\nolinkurl{pschloss@umich.edu}};
\href{mailto:akhagan@alliancescc.com}{\nolinkurl{akhagan@alliancescc.com}}

*Present address: Alliance SciComm \& Consulting, LLC, Linden, MI

1. Department of Microbiology and Immunology, University of Michigan,
Ann Arbor, MI

2. Department of Biology, University of Akron, Akron, OH

\newpage
\linenumbers

\hypertarget{abstract}{%
\subsection{Abstract}\label{abstract}}

Despite 50\% of biology Ph.D.~graduates being women, the number of women
that advance in academia decreases at each level (e.g.~from graduate to
post-doctorate to tenure-track). Recently, scientific societies and
publishers have begun examining internal submissions data to evaluate
representation and evaluation of women in their peer review processes;
however, representation and attitudes differ by scientific field and no
studies to-date have investigated academic publishing in the field of
microbiology. Using manuscripts submitted between January 2012 and
August 2018 to the 15 journals published by the American Society for
Microbiology (ASM), we describe the representation of women at ASM
journals and the outcomes of their manuscripts. Senior women authors at
ASM journals were underrepresented compared to global and society
estimates of microbiology researchers. Additionally, manuscripts
submitted by corresponding authors that were women received more
negative outcomes than those submitted by men. These negative outcomes
were somewhat mediated by whether or not the corresponding author was
based in the US, and by the type of institution for US-based authors.
Nonetheless, the pattern for women corresponding authors to receive more
negative outcomes on their submitted manuscripts held. We conclude with
suggestions to improve the representation of and decrease structural
penalties against women.

\hypertarget{importance}{%
\subsection{Importance}\label{importance}}

Barriers in science and academia have prevented women from becoming
researchers and experts that are viewed as equivalent to their
colleagues who are men. We evaluated the participation and success of
women researchers at ASM journals to better understand their success in
the field of microbiology. We found that women are underrepresented as
expert scientists at ASM journals. This is, in part, due to a
combination of both low submissions from senior women authors and more
negative outcomes on submitted manuscripts for women compared to men.

\hypertarget{introduction}{%
\subsection{Introduction}\label{introduction}}

Evidence has accumulated over the decades that academic research has a
representation problem. While at least 50\% of biology Ph.D.~graduates
are women, the number of women in postdoctoral positions and
tenure-track positions are less than 40 and 30\%, respectively (1).
There have been many proposed reasons for these disparities, which
include biases in training and hiring, the impact of children on career
trajectories, a lack of support for primary caregivers, a lack of
recognition, lower perceived competency, and less productivity as
measured by research publications (1--8). These issues do not act
independent of one another, instead they accumulate for both individuals
and the community, much as advantages do (9--11). Accordingly,
addressing these issues necessitates multi-level approaches from all
institutions and members of the scientific community.

Scientific societies play an integral role in the formation and
maintenance of scientific communities--they host conferences that
provide forums for knowledge exchange, networking, and opportunities for
increased visibility as a researcher. Scientific societies also
frequently publish the most reputable journals in their field,
facilitating the peer review process to vet new research submissions
(12). Recently, scientific societies and publishers have begun examining
internal submissions data to evaluate representation of and bias against
women in their peer review processes. The American Geological Union
found that while the acceptance rate of women-authored publications was
greater than publications authored by men, women submitted fewer
manuscripts than men and were used as reviewers only 20\% of the time
(13), a factor that is reported to be influenced by the gender of the
editor (14). Several studies have concluded that there is no significant
bias against papers authored by women (14--19). Recent reports of
manuscript outcomes at publishers for ecology and evolution, physics,
and chemistry journals have found that women-authored papers are less
likely to have positive peer reviews and outcomes (20--23).

The representation of women scientists and gender attitudes differ by
scientific field and no studies to-date have investigated academic
publishing in the field of microbiology. The American Society for
Microbiology (ASM) is one of the largest life science societies, with an
average membership of 41,000 since 1990. A recent statement notes that
``A diverse ASM enhances the microbial sciences, increases innovation,
strengthens the community and sustains the profession'' and pledges to
``address all members' needs through development and assessment of
programs and services'' that aims to ensure ``equitable access and
accountability through transparent procedures and communication'' (24).
One of the ASM's services is the publication of microbiology research
through a suite of research and review journals. Between January 2012
and August 2018, ASM published 25,818 original research papers across 15
different journals: \emph{Antimicrobial Agents and Chemotherapy} (AAC),
\emph{Applied and Environmental Microbiology} (AEM), \emph{Clinical and
Vaccine Immunology} (CVI), \emph{Clinical Microbiology Reviews} (CMR),
\emph{Eukaryotic Cell} (EC), \emph{Infection and Immunity} (IAI),
\emph{Journal of Bacteriology} (JB), \emph{Journal of Clinical
Microbiology} (JCM), \emph{Journal of Virology} (JVI), \emph{mBio},
\emph{Microbiology and Molecular Biology Reviews} (MMBR), \emph{Genome
Announcements} (GA, now \emph{Microbiology Resource Announcements}),
\emph{Molecular and Cellular Biology} (MCB), \emph{mSphere}, and
\emph{mSystems}. Two journals, EC and CVI, were retired during the
period under study and three journals, GA/MRA, MMBR, and CMR, were
excluded from the analysis due to their relatively low number of
submissions. The goal of our research study was to describe the
population of the ASM journals both through the gender-based
representation of authors, reviewers, and editors and the associated
peer review outcomes.

\hypertarget{results}{%
\subsection{Results}\label{results}}

Over 100,000 manuscript records were obtained for the period between
January 2012 and August 2018 (Fig. 1). Each of these were evaluated by
editors and some by reviewers, leading to multiple possible outcomes. At
the ASM journals, manuscripts may be immediately rejected by editors
instead of being sent to peer review, often due to issues of scope or
quality. These were defined as editorial rejections and identified as
manuscripts rejected without review. Alternately, editors send a
majority of manuscripts out for review by two or more experts in the
field selected from a list of potential reviewers suggested by the
authors and/or editors. Reviewers give feedback to the authors and
editor, who decides whether the manuscript in question should be
accepted, rejected, or sent back for revision. Manuscripts with
suggested revisions that are expected to take more than 30 days to
address are rejected, but generally encouraged to resubmit. If
resubmitted, the authors are asked to note the previous manuscript and
the re-submission is assigned a new manuscript number. Multiple related
manuscripts were tracked together by generating a unique grouped
manuscript number based on the recorded related manuscript numbers. This
grouped manuscript number served dual purposes of tracking a single
manuscript through multiple rejections and avoiding duplicate counts of
authors for a single manuscript. After eliminating non-primary research
manuscripts and linking records for resubmitted manuscripts, we
identified 79,189 unique manuscripts (Fig. 1).

We inferred the gender of both the peer review participants (e.g.,
editor-in-chief, editors, reviewers) and authors on the manuscripts
evaluated during this time period using a social media-informed
classification algorithm with stringent criteria and validation process
(Supp Text, Fig. S1). We recognize that biological sex (male/female) is
not always equivalent to the gender that an individual presents as
(man/woman), which is also distinct from the gender(s) that an
individual may self-identify as. For the purposes of this manuscript, we
choose to focus on the presenting gender based on first names (and
appearance for editors), as this information is what reviewers and
editors also have available. The sensitivity, specificity, and accuracy
of our method were 0.97 (maximum of 1.0) when validated against a
curated set of authors (Table S1). The accuracy was 0.99 when applied to
the list of editors, whose genders were inferred by hand using Google
(Supp Text). In addition to identifying journal participants as men or
women, this method of gender inference resulted in a category of
individuals whose gender could not be reliably inferred (i.e., unknown).
We included those individuals whose names did not allow a high degree of
confidence for gender inference in the ``unknown'' category of our
analysis, which is shown in many of the plots depicting representation
of the population. These individuals were not included in the comparison
of manuscript outcomes. Finally, we refer to editors and peer reviewers
collectively as gatekeepers, which describes and recognizes their
essential role in maintaining the scientific quality of manuscripts
accepted (or rejected) at peer reviewed journals (25, 26).

\textbf{Men dominated as gatekeepers and senior authors.} We first
evaluated the representation of men and women who were gatekeepers
during the study period. Each journal is led by an editor-in-chief (EIC)
who manages journal scope and quality standards through a board of
editors with field expertise that, in turn, handle the peer review
process. There were 17 EICs, 17.6\% of which were women. Four years
before retirement, the EIC of CVI transferred from a man to a woman,
while JVI has had a woman as EIC since 2012. The total number of editors
at all ASM journals combined over the duration of our study (senior
editors and editors pooled) was 1015, 28.8\% of which were women.

Over 40\% of both men and women editors were from US-based R1
institutions, defined as doctoral-granting universities with very high
research activity (27). Non-US institutions and US medical schools or
research institutions supplied the next largest proportions of editors
(Fig. 2A)(27). Since 2012, there was a slow trend toward equivalent
gender representation among editors (Fig. 2B). Individual journal trends
varied considerably, though most had slow trends toward parity (Fig.
2C). CVI and \emph{mSphere} were the only ASM journals to have
accomplished equivalent representation of men and women, with CVI having
a greater proportion of women editors than men before it was retired. EC
was the only journal with an increasing parity gap.

Altogether, 30439 reviewers submitted reviews and 24.6\% were inferred
to be women. The greatest proportion of reviewers (over 50\% of all
groups) came from non-US institutions, while R1 institutions supplied
the next largest cohort of reviewers (Fig. 2D). The proportions of each
gender group were consistent over time among reviewers at the ASM
journals (Fig. 2E) and were representative of both the suggested
reviewers at all journals combined, and the actual reviewer proportions
at most journals (Fig. S2).

\textbf{Editorial workloads were not proportionate.} To evaluate the
editorial workload for each gender, we calculated the proportion of
manuscripts handled by editors of each gender (excluding editorial
rejections), relative to their representation. If the workload is
proportionate, then the workload for each gender will be equivalent to
the gender's representation at that journal. Across all of the journals
combined, men handled a slightly greater proportion of manuscripts than
women, relative to their respective editorial representations (Fig. 3A).
This trend was present at most journals with varying degrees of
difference between workload and representation (Fig. 2C). For instance,
at \emph{mSphere}, both workload and representation were identical;
however, CVI, \emph{mBio}, and JVI each had periods at which the
workload for women editors was much higher than their representation,
with corresponding decreases in the workload of men. In the years
preceding its retirement, the representation of women at CVI increased,
decreasing the gap in editorial workload. However, representation and
relative workloads for men and women editors at JVI held steady over
time, while the proportional workload for women at \emph{mBio} has
increased.

The median number of manuscripts reviewed by men, women, and unknown
gendered individuals was 2, for each group. Half of those in the men,
women, or unknown gender groups reviewed between one and 5, 4, or 3
manuscripts each, respectively (Fig. 3B). Conversely, 44.6\% of men,
40.1\% of women, and 48.6\% of unknown gendered reviewers reviewed only
one manuscript, suggesting that women were more likely than other groups
to review multiple manuscripts. Reviewers of all gender groups accepted
fewer requests to review from women editors (average of 47.8\%) than
from men (average of 53.3\%; Fig. 3C). Reviewers were also less likely
to respond to women editors than men (no response rate averages of 25.1
and 19.9\%, respectively). Both men and women editors contacted
reviewers from all three gender groups in similar proportions, with
women editors contacting 76.4\% of suggested reviewers and men
contacting 74.1\% (median of the percent contacted from each gender
group).

\textbf{Women were underrepresented as authors.} Globally, microbiology
researchers are 60\% men and 40\% women (28). In September 2018, 38.4\%
of ASM members who reported their gender were women. We wanted to
determine if these proportions were similar for senior authors at the
journals and to understand the distribution of each gender group among
submitted manuscripts and published papers. We began by describing
senior author (last/corresponding author) institutions by gender group.
Over 60\% of submitting senior authors were from non-US institutions,
followed by about 20\% from R1 institutions. The proportion of
manuscripts submitted from US institutions by women senior authors was
31\% versus 36\% from women who were senior authors at non-US
institutions. Women senior authors were more highly represented at low
research universities and federal research institutions than at any
other US-based institution (Fig. 4A). The proportions of all men and
women (senior and co-) authors at the ASM journals decreased over time
at equivalent rates, while the proportion of unknown gendered authors
increased; the ratio of men to women authors was 4 to 3 (i.e., 57\% men;
Fig. 4B).

In the field of microbiology, order of authorship on a manuscript
signals the type and magnitude of contributions to the finished product.
First and last authorship are the most prestigious. First authors are
generally trainees (e.g., students or post-docs) or early career
researchers responsible for performing the bulk of the project, while
last authors are generally lead investigators that supplied conceptual
guidance and resources to complete the project. Middle authors are
generally responsible for technical analyses and methods. Any author can
also be a corresponding author, which we identified as the individual
responsible for communicating with publishing staff during peer review
(as opposed to an author to whom readers direct questions), of which
there can be multiple.

The proportion of manuscripts submitted with men or women as first
authors remained constant at 29.1 and 30.7\%, respectively (Fig. 4C,
dashed). The proportions of first author published papers were nearly
identical at 33.1\% for men and 33.8\% for women (Fig. 4C, solid). The
proportion of submitted manuscripts with men corresponding authors
remained steady at an average of 41.6\% and the proportion with women
corresponding authors was 23.4\% (Fig. 4D, dashed); the proportion of
published unknown gender authors declined. Both men and women
corresponding authors had a greater proportion of papers published than
manuscripts submitted. Accordingly, manuscripts with corresponding
authors of unknown gender were rejected at a higher rate than their
submission. The difference between the percent of submitted manuscripts
and published papers was 8.2\% when men were corresponding authors, but
only 0.9\% when women were corresponding authors, making the submitted
to published difference near equal (Fig. 4C, solid). This trend was
similar for middle and last authors (Fig. S3).

Of the 38594 multi-author manuscripts submitted by men corresponding
authors, 23.5\% had zero authors inferred to be women. In contrast, 7253
(36.3\%) of the manuscripts submitted by women corresponding authors had
more than half of the authors inferred as women, exceeding those
submitted by men corresponding authors in both the number (3247) and
percent (8.4) of submissions. Additionally, the proportion of women
authors decreased as the number of authors increased, such that when the
number of authors exceeded 30 on a manuscript (N=59), the proportion of
individuals inferred to be women was always below 51\% (Fig. S4). Men
submitted 225 single-authored manuscripts while women submitted 69
single-authored manuscripts.

We hypothesized that we would be able to predict the inferred gender of
the corresponding author using a logistic regression model trained on
the following variables: whether the corresponding author's institution
was in the U.S., the total number of authors, the proportion of authors
that were women, whether the paper was published, the gender of senior
editors and editors, the number of revisions, and whether the manuscript
was editorially rejected at any point. We measured the model's
performance using the area under the receiver operating characteristic
curve (AUROC). The AUROC value is a predictive performance metric that
ranges from 0.0, where the model's predictions are completely wrong, to
1.0, where the model distinguishes perfectly between outcomes. A value
of 0.5 indicates that the model did not perform better than a random
assignment. The median AUROC value of our model to predict the
corresponding author's inferred gender was 0.7 (Fig. S5A, panel A). The
variable with the largest absolute weight (i.e., the most predictive
value), in our model was the proportion of women authors (Fig. S5C).
These results indicate that manuscript submission data was capable of
predicting the inferred gender of the corresponding author, but that the
prediction was primarily driven by the percentage of authors that were
inferred to be women.

As described above, first authors were slightly more likely to be women
(30.7\%W vs 29.1\%M), but corresponding authors were significantly more
likely to be men (23.44\%W vs 41.59\%M). A concern is that if authors
are not retained to transition from junior to senior status, they will
be left out of the gatekeeping roles. Since authorship conventions
indicate that last and corresponding authors are typically senior
authors, we combined both first and middle authors into the ``junior''
author role and used the unique identifiers assigned to each account to
track individuals through the possible roles at the ASM journals. There
were 75451 women who participated as junior authors (first/middle) at
the ASM journals. Of those junior authors who were women, 8.2\% also
participated as senior authors (last/corresponding), 8.9\% were
potential reviewers and 5.4\% participated as reviewers. 0.2\% of women
junior authors became editors at the ASM journals over the 6 year period
studied. For men, there were a total of 83727 junior authors, where
13.6\% also participated as senior authors, 16.7\% were potential
reviewers, and 11.1\% actually reviewed. 0.7\% of men junior authors
became editors at the ASM journals. Overall, women who participated at
the ASM journals as junior authors were half as likely to move to senior
author or reviewer roles, and 30\% as likely to be an editor than men at
the ASM journals.

\textbf{Manuscripts submitted by women have more negative outcomes than
those submitted by men.} To further investigate the difference in
percents of published and submitted proportions for men and women
authors (Fig. 4CD, Fig. S3), we compared the rejection rates of men and
women at each author stage (first, middle, corresponding, and last). To
more easily visualize and understand the differences in outcomes
according to author gender, we calculated the outcome rate for each
gender then subtracted the rate for women from men to generate the
percentage point difference. To correct for the disparity in
participation by women compared to men, all percentage point comparisons
were made relative to the gender and population in question. Where men
received an outcome more often than women, the value was positive (blue)
while a negative value (orange) indicated that women outperformed men in
the given metric. For the following analyses, only manuscripts authored
by an individual inferred to be a man or woman were included. Finally,
these analyses were conducted on all available manuscripts, not a
statistical sampling. As a result, statistical tests were only required
for correlative analyses.

Middle authors were rejected at equivalent rates for men and women (a
0.23 percentage point difference across all journals). However,
manuscripts with senior women authors were rejected more frequently than
those authored by men with -6.7 and -6.0 percentage point differences
for corresponding and last authors, respectively (Fig. 5A, vertical
lines). The overall trend of increased rejection for women was most
pronounced at MCB, JB, IAI and AAC. The greatest differences were
observed when comparing the outcome of corresponding authors by gender,
so we used this sub-population to further examine the difference in
manuscript acceptance and rejection rates between men and women.

We next compared the rejection rates for men and women corresponding
authors after two review points, initial editor review and the first
round of peer review. Manuscripts authored by women were editorially
rejected by as much as 12 percentage points more often than those
authored by men (Fig. 5B). The difference at all of the ASM journals
combined favored men by -3.8 percentage points (vertical line). MCB and
\emph{mBio} had the most extreme percentage point differences.
Manuscripts authored by men and women were equally likely to be accepted
after the first round of review (Fig. 5C, right panel). However,
women-authored papers were rejected (left panel) more often than men
by5.6 percentage points. Meanwhile, men-authored papers were given
revision (center panel) decisions -5.6 percentage points more frequently
than women (Fig. 5C, vertical lines). JB, AAC, and MCB had the most
extreme differences for rejection and revision decisions. Percentage
point differences were not correlated with journal prestige as measured
by 2018 impact factors (R\({^2}\) = -0.022, P = 0.787).

In addition to manuscript decisions, other disparate outcomes may occur
during the peer review process (29). To determine whether accepted
women-authored manuscripts spent more time between being submitted and
being ready for publication, we compared the number of revisions, days
spent in the ASM peer review system, and the number of days between
submission and being ready for publication to those authored by men.
Manuscripts authored by women took slightly longer to complete than
those by men at all journals, an additional 1 to 9 days on average from
submission to ready for publication (Fig.S6A). This was despite spending
similar amounts of time in the ASM journal peer review system (from 1
day less to 4 more than men) (Fig. S6B) and having the same median
number of revisions prior to acceptance (Median = 2, IQR = 0).

To understand how a gatekeeper's (editor/reviewer) gender interacted
with decision types (e.g., Fig. 5C), we grouped editor decisions and
reviewer suggestions according to the gatekeeper's inferred gender
(unknowns excluded). Both men and women editors rejected proportionally
more women-authored papers, however the percentage point difference in
decisions were slightly larger for men-edited manuscripts (Fig. 6A).
Reviewers were more likely to suggest rejection for women-authored
manuscripts as compared to men and a minimal difference in revise
recommendations was observed (Fig. 6B). Both men and women reviewers
recommended rejection more often for women-authored manuscripts although
men recommended acceptance and revision more frequently for men-authored
manuscripts than women did (Fig. 6C).

To evaluate if inferred gender played a role in manuscript editorial
decisions, we trained a logistic regression model to predict whether a
manuscript was reviewed (i.e., editorially rejected or not). We used the
inferred genders of the senior editor, editor, and corresponding author,
as well as the proportion of authors that were women as variables to
train the model (Fig. S5B). The median AUROC value was 0.61 (Fig. S5A,
panel B), which indicated that editorial decisions were not random,
however, the relatively low AUROC value indicated that there are factors
not included in our model that influence editorial decisions.

\textbf{Multiple factors contribute to the overperformance of men.} The
association between inferred gender and manuscript decision could be
attributed to implicit gender bias by journal gatekeepers, however,
there are other types of bias that may contribute to, or obscure, gender
bias; for instance, a recent evaluation of peer-review outcomes at
\emph{eLife} found evidence of preference for research submitted by
authors from a gatekeeper's own country or region (20). Other studies
have documented prestige bias, where men are over-represented in more
prestigious (i.e., more respected and selective) programs (30). It is
therefore possible, that what seems to be gender bias could be
geographic or prestige bias interacting with the increased proportion of
women submitting from outside the US or from lower prestige institutions
(e.g., the highest rate of submissions from women were at low research
institutions, 37\%; Fig. 4A).

To quantify how these factors affected manuscript decisions, we next
looked at the outcome of manuscripts submitted only by corresponding
authors at US institutions, because these institutions represented the
majority of manuscripts and could be classified by using the Carnegie
Classification of Institutions of Higher Education (27). We used the
same strategy as described above. When only considering US-based
authors, the percentage point difference for editorial rejections
increased from -3.8 to -1.4 percentage points, reducing the bias against
papers submitted by women (Fig. 7A). The trend of percentage point
difference in decisions after review for US-based authors mirrored those
seen for all corresponding authors at the journal level (Fig. 7B). The
over-representation of women in rejection decisions increased from -5.6
to -4.4 percentage points, and the over-representation of men in revise
only decisions decreased from 5.6 to 4.2, moving manuscript outcomes
toward parity (Fig. 7B). The difference in the rate of accept decisions
changed from -1.4 to 0.2 percentage points after restricting the
analysis to US-based authors, indicating near equal acceptance for
corresponding authors of both genders. These results suggest that the
country of origin (i.e., US versus not) accounted for some of the
differences in outcomes by inferred gender, particularly for editorial
rejections.

To address institution-based prestige bias, we split the US-based
corresponding authors according to the type of institution they were
affiliated with (based on the Carnegie classification) and re-evaluated
the differences for men and women (27). Editorial rejections occurred
most often for women from medical schools or institutes, followed by
those from R2 institutions: 32\% and 28\% of manuscripts from each
institution were submitted by women, respectively (Fig. 7C, Fig. S7A).
This percentage point difference in the editorial rejections of
corresponding authors from medical schools or institutes was spread
across most of the ASM journals, while the editorial rejection of papers
submitted from women at R2 institutions was driven primarily by
submissions to JCM (Fig. S7A). Evaluating the percentage point
difference in acceptance rates by institution and inferred gender
mirrored that of editorial rejections for some journals, where
submissions from men recieved better outcomes than submissions from
women (Fig. 7CD and S7BC). For instance, manuscripts submitted by men
from medical schools or institutes were accepted up to 10 percentage
points more often than those submitted by women (Fig. 7C).

To evaluate if these factors affect manuscript decisions, we trained a
logistic regression model to predict whether a manuscript was
editorially rejected using the variables: origin (US vs non),
institution (US institution type), number of authors, proportion of
authors that were women, and the inferred genders of both gatekeepers
and corresponding authors. The model had a median AUROC value of 0.67
(Fig. S5A, panel C), which indicated a non-random interaction between
these factors and editorial decisions. Manuscripts from authors at U.S.
``other'' institutions, men EICs, men that were corresponding authors
from ``other'' U.S. institutions, and women from medical schools and
institutes were all more associated with editorial rejections (Fig.
S7D). Conversely, manuscripts from R1 institutions, authors from the
U.S., EICs that were women, and the number of authors were all more
likely to be associated with review (Fig. S7D). These results confirm
that the country of origin and class of institution impact decisions in
a non-random manner, though not as much as gender.

A final factor we considered was whether the type of research pursued by
men as opposed to women may impact manuscript outcomes. Black women
philosophers and physicists have described the devaluation of
non-traditional sub-disciplines in their fields (31--33). This concept
originally described bias against Black women---the intersection of two
historically marginalized identities. However, the idea that researchers
in an established core field might be skeptical of less established, or
non-traditional, sub-field research likely applies elsewhere. The
disparate outcomes of sub-fields in a gendered context has recently been
observed in the biomedical sciences, where NIH proposals focusing on
womens' reproductive health were the least likely to be funded (34). To
explore this phenomenon in the ASM journals, we looked at the editorial
rejection rates of manuscripts (regardless of origin or institution) for
each research category at the five largest ASM journals: AAC, AEM, IAI,
JVI, and JCM. Together, these journals account for 47\% of the
manuscripts analyzed in this study and comprise 55 categories.

The number of submissions in each category ranged from 1 (``FDA
Approval'' at AAC) to 2952 (``Bacteriology'' at JCM) while the
acceptance rates varied from 29.4\% (``Chemistry:Biosynthesis'' at AAC)
to 71.3\% (``Structure and Assembly'' at JVI) (Table 1). We argued that
the number of submissions to each category could help indicate core
versus periphery subfields, (i.e., core subfields would have more
submissions than periphery subfields) and based on the literature
to-date, we expected that periphery subfields might have a higher
participation of women (31--33). Women submitted on average 35.3\% of
the manuscripts to each category, ranging from 20\% to 86\% (Table 1).
There was not a correlation between the proportion of women authors and
the number of submissions (R\({^2}\) = -0.0177, P = 0.779) to each
category. Nor was there a correlation between the proportion of women
authors and the category acceptance rate (R\({^2}\) = 0.041, P = 0.078).
These data suggest that there was not a relationship between the
participation of women and either the number of submissions or the
acceptance rate of categories in our dataset.

We next looked at the percentage point differences in performance for
men and women in each category at two decision points: editorial
rejection and rejection after the first review. Each journal focuses on
a different facet of microbiology or immunology, making the results
difficult to compare directly. However, the pattern of increased
rejection rates for women was maintained across most categories with
some displaying major differences in gendered performance (Fig. S8). For
instance, the ``Biologic Response Modifier'' (e.g., immunotherapy)
sub-category at AAC, had extreme differences for both editorial
rejections and rejections after review where men were favored by 30 and
40 percentage points, respectively. While that category had a relatively
low number of submissions (N = 44), 43\% were from women (Fig. S8A).
``Mycology'', was a category at two journals, AEM and JCM. At both
journals, men received favorable outcomes relative to women in this
category. At AEM, there were 73 ``Mycology'' submissions, 44\% from
women authors with an almost 20 percentage point difference favoring the
editorial rejection outcomes of men corresponding authors. Men authors
were slightly less favored in rejections after review at a 10 percentage
point difference (Fig. S8B). JCM had 587 ``Mycology'' submissions with a
submission rate of 39\% from women authors (Fig. S8D). Differences
between JCM ``Mycology'' outcomes also favored men authors by almost 10
and 12 percentage points for editorial rejections and rejections after
review, respectively.

Because of these extreme percentage point differences in categories with
high women authorship, we next asked if the number of women
participating in a particular category was related to manuscript
outcomes. There was no correlation between the difference in editorial
rejection by category and the percent of women that were either authors
(R\({^2}\) = -0.003, P = 0.363) or editors (R\({^2}\) = -0.018, P =
0.765). The percent of women authors and percent of women editors in
journal categories did not correlate either (R\({^2}\) = -0.007, P =
0.682), which is likely related to the underrepresentation of women
editors in categories dominated by women authors (e.g.,
``Epidemiology''). These data suggest the possibility of persistent
negative outcomes against women in particular fields (e.g.,
``Mycology''), though it does not seem to relate to either the number of
submissions or participation of women in those subfields.

\hypertarget{discussion}{%
\subsection{Discussion}\label{discussion}}

We described the representation of inferred men and women participating
in the submission and peer review process at the ASM journals between
January 2012 and August 2018 and compared editorial outcomes according
to the authors' inferred gender. Women were consistently
under-represented (30\% or less in all levels of the peer review
process) excluding first authors, where women represented about 50\% of
authors where we could infer a gender (Figs. 2 and 4). Women and men
editors had proportionate workloads across all of the ASM journals
combined, but those workloads were disproportionate at the journal level
and the overburdened gender varied by journal (Figs. 2 and 3).
Additionally, manuscripts submitted by women corresponding authors
received more negative outcomes (e.g., editorial rejections) than those
submitted by men (Figs. 5 and 6). These negative outcomes were somewhat
mediated by whether the corresponding author was based in the US, the
type of institution for US-based authors, and the research category
(Figs. 7 and S8). However, the trend for women corresponding authors to
receive more negative outcomes held across all analyses, indicating a
pattern of gender-influenced editorial decisions regardless of journal
prestige (as determined by impact factor). Together, these data indicate
a persistent penalty for senior women microbiologists who participate at
the ASM journals.

How to define representation and determine what the leadership should
look like are recurring questions in STEM. Ideally, the representation
for men and women corresponding authors, reviewers, and editors would
reflect the number of Ph.D.s awarded (about 50\% each, when considered
on a binary spectrum). We argue that the goal should depend on the
workload and visibility of the position. Since high visibility positions
(e.g., editor, EIC) are filled by a smaller number of individuals that
are responsible for recruiting more individuals into leadership, filling
these positions should be done aspirationally (i.e., 50\% should be
women if the goal were an aspirational leadership). This allows greater
visibility for women as experts, expansion of the potential reviewer
network, and recruitment into those positions (35--37). Conversely,
lower visibility positions (e.g., reviewers) require effort from a
greater number of individuals and should thus be representational of the
field to avoid overburdening the minority population (i.e., since 23.5\%
of corresponding authors at the ASM journals are women, then 20-25\% of
reviewers should be women). Balancing the workload is particularly
important given the literature indicating that women faculty have higher
institutional service loads than their counterparts who are men (38).

Our data also revealed some disturbing patterns in gendered authorship
that have implications for the retention of women microbiologists.
Previous research suggests that women who collaborate with other women
receive less credit for these publications than when they collaborate
with men (39), and that women are more likely to yield corresponding
authorship to colleagues that are men (21). In our linear regression
models, the number of authors on a manuscript was the largest
contributor to avoiding editorial rejections, suggesting that highly
collaborative research is preferred by editors (40). This observation
was supported by the positive correlation between citations and author
count (Fig. S7). Thus it concerns us that when the number of authors
exceeded 30 on a manuscript (N=59), the proportion of individuals
inferred to be women was always below 51\%, despite equivalent numbers
of trainees in the biological sciences (Fig. S4). And while women
corresponding authors submitted fewer manuscripts, more of them (both
numerically and proportionally), had a majority of co-authors inferred
to be women, compared to those submitted by men corresponding authors.
These data support previous findings that women are more likely to
collaborate with other women (23, 41--43). Additionally, the proportion
of women authors was the greatest predictor of corresponding author
gender. This gender-based segregation of collaborations at the ASM
journals likely has had consequences in pay and promotion for women
microbiologists and could be a factor in the decreased retention of
senior women. We predict that the low retention is aggravated by the
under representation of women as corresponding authors, which also has
negative consequences for both their careers and microbiology. Since
senior authorships impact status, visibility, and salary, the under
representation of women as senior authors and reviewers likely hampers
their career progression and desire to progress (18, 44). The retention
of women (and other marginalized groups) is important to the progress of
microbiology since less diversity in science limits the diversity of
perspectives and approaches, thus stunting the search for knowledge.

Even if a gatekeeper does not know the corresponding author or their
gender, there remain ample avenues for implicit bias during peer review.
The stricter standard of competency has led women to adopt different
writing styles from men, resulting in manuscripts with increased
explanations, detail, and readability than those authored by men (29,
45). Additionally, women are often at a disadvantage for the resources
required for highly competitive fields due to cumulative penalties
(9--11). As a result, corresponding authors that are women may be
spending their resources in research fields where competition impacts
are mitigated and/or on topics that are historically understudied, thus
these are cues to gender and perceived competency (31--33).
Alternatively, non-traditional research may be seen as less impactful,
leading to poorer peer review outcomes (34). These possibilities are
reflected in our data, since while the number of revisions before
publication is identical for both men and women, manuscripts authored by
women have increased rejection rates and time spent on revision. This
suggests that manuscripts submitted by women receive more involved
critiques (i.e., work) from reviewers and/or their competency to
complete revisions within the prescribed 30 days is doubted, compared to
men. Women may also feel that they need to do more to meet reviewer
expectations, thus leading to longer periods between a decision and
resubmission. Finally, our data show a penalty for women researching
mycology (Fig. S8). Despite being among the most deadly infectious
diseases in 2016 (along with tuberculosis and diarrheal diseases),
mycology is an underserved, and underfunded, field in microbiology that
has historically been considered unimportant (46--49). Microbiology
would benefit from a more nuanced evaluation of sub-fields to better
understand how they interact with gender and peer review outcomes.

A limitation to our methodology is the use of an algorithm to infer
gender from first names. While we report a high accuracy (0.97-0.99)
where gender was inferred, this method left us with a category of
unknown gendered individuals. Additionally, the gender of an individual
may be interpreted differently according to the reader (e.g., Kim is
predominately a woman's name in the U.S., but likely a man's name in
other cultures). The increase in unknown gendered authors corresponds to
an increase in submissions to the ASM journals from Asian countries,
particularly China. Anecdotally, most editorial rejections are poor
quality papers from Asia, and our method had low performance on
non-gendered languages from this region (Supp Text, Fig. S1), thus
excluding many Asia-submitted manuscripts and increasing our confidence
that the trends observed were gender-based. For corresponding authors,
manuscript submissions are the end product of several other prior
decisions such as a mentor's implicit bias(es), postdoctoral
fellowships, faculty applications, start-up funding negotiations, and
grant proposals. These prior factors, which cannot be accounted for in
our analysis, along with the small effect size observed in some
analyses, limit quantifying the role of gatekeeper decisions in the
disparate gender-influenced outcomes. However, the consistency of
decisions to benefit men corresponding authors over women across all of
the journals included in this study, in addition to accumulated
literature to-date, confirms that this descriptive study is highly
relevant for the ASM as a society. Our findings offer opportunities to
address gendered representation in microbiology and systemic barriers in
peer review at our journals.

All parties have an opportunity and obligation to advance marginalized
groups in science (50, 51). We suggest that journals develop a visible
mission, vision, or other statement that commits to equity, justice, and
inclusion and includes a non-discrimination clause regarding decisions
made by editors and editors-in-chief. This non-discrimination clause
should be backed by a specific protocol for the reporting of, and
response to, instances of discrimination and harassment. Second, society
journals should begin collecting additional data from authors and
gatekeepers such as race, ethnicity, gender identity, and disabilities.
These data should not be available to journal gatekeepers but instead
kept in a dis-aggregated manner that allows for public presentation,
tracking the success of inclusive measures, and to maintain
accountability. Third, society journals can implement mechanisms to
explicitly provide support for women and other marginalized groups,
reward inclusive behavior by gatekeepers, nominate more women to
leadership positions, and recruit manuscripts from sub-fields that are
more likely to attract women and other marginalized groups (34). We can
all help advance women (and other marginalized groups) within the peer
review system by changing how we select experts in our field. For
instance, authors can suggest more women as reviewers using
``Diversify'' resources (52), while reviewers can agree to review for
women editors more often. Editors can rely more on manuscript reference
lists and data base searches than personal knowledge to recruit
reviewers (53), and journals can improve the interactivity and
functionality of the reviewer selection software. Given the propensity
for journals to recruit editors and EiCs from within their already
skewed reviewer pools, opening searches to include more scientists in
their reviewer pool and/or editors from outside the journal while
enacting more transparent processes could be a major component of
improving representation. Growing evidence suggests that representation
problems in STEM are due to retention rather than recruitment. We need
to align journal practices to foster the retention of women and other
marginalized groups.

Most approaches to disparate outcomes focus on choices made by
individuals, such as double-blinded reviews and implicit bias training.
These cannot fully remedy the effects of implicit bias and may even
worsen outcomes (54, 55). Since disparate outcomes (by gender,
geographic, prestige, or otherwise) are primarily the result of
accumulated disadvantages and actions resulting from implicit biases and
systemic ``-isms'', a structural, system-wide approach is required
(56--58). Broadly, peer review is a nebulous process with expectations
and outcomes that vary considerably, even within a single journal.
Academic writing courses suffer similar issues and have sought to remedy
them with rubrics. When implemented correctly, rubrics can reduce
implicit bias during evaluation and enhance the evaluation process for
both the evaluator and the evaluatee (59--62). We argue that rubrics
could be implemented in the peer review process to focus reviewer
comments, clarify editorial decisions, and improve the author
experience. Such rubrics should increase the emphasis on solid research,
as opposed to novel or ``impactful'' research, the latter of which is a
highly subjective measure (63, 64). This might also change the overall
negative attitude toward replicative research and negative results, thus
bolstering the field through reproducibility. We also argue that
reconsidering journal scope and the membership of honorary editorial
boards might help address structural penalties resulting from implicit
bias against women (and other marginalized groups) in peer review.
Expanding journal scope and adding more handling editors would improve
the breadth of research published, thus providing a home for more
non-traditional and underserved research fields (the case at
\emph{mSphere} with an increased pool of editors). Implementing these
steps to decrease implicit bias and structural penalties---review
rubrics, increased focus on solid research, expansion of journal scopes
and editorial boards---will also standardize competency principles for
researchers at the ASM journals and improve microbiology as a whole.

Although the level of bias at many of the ASM journals is small, it is
present. Peer review at the ASM journals is not immune to the
accumulated disadvantages against women in microbiology. However, the
adaptation of women and other marginalized groups to implicit bias
(e.g., area of research and communication styles), make it impossible to
address at the individual level. Instead, we must commit to changing the
fundamental structure and goals of peer review to minimize the impact of
such bias. We encourage the ASM journals, as well as other societies, to
institute more fair and transparent procedures and approaches of peer
review. The self-correcting nature of science is a badge that scientists
wear proudly, but no single report or action can correct the inertia of
a centuries-old institution. Instead, it requires the long-standing and
steady actions of many. Our findings reflect many similar reports, and
suggest concrete actions to correct the inertia of peer review at all
levels. The next step is commitment and implementation.

\hypertarget{data-and-methods}{%
\subsection{Data and Methods}\label{data-and-methods}}

\textbf{Data.} All manuscripts handled by the ASM journals (e.g.,
\emph{mBio}, \emph{Journal of Virology}) that received an editorial
decision between January 1, 2012 and August 31, 2018 were supplied as
XML files by the ASM's publishing platform, eJP. Data were extracted
from the XML documents provided, manipulated, and visualized using R
statistical software (version 3.4.4) and relevant packages. Variables of
interest included: the manuscript number assigned to each submission,
manuscript type (e.g., full length research, erratum, editorial),
category (e.g., microbial ecology), related (i.e., previously submitted)
manuscripts, number of versions submitted, dates (e.g., submission,
decision), author data (e.g., first, last, and corresponding authorship,
total number of authors), reviewer data (e.g., recommendation, editor
decision), and personal data (names, institutions, country) of the
editors, authors, and reviewers. Since reviews and commentaries are
often commissioned, only original, research-based manuscripts were
included in this analysis, e.g., long- and short-form research articles,
New-Data Letters, Observations, Opinion/Hypothesis articles, and
Fast-Track Communications. To help protect the confidentiality of peer
review, names were removed from all records, and identifying data (e.g.,
manuscript numbers, days of date) were replaced with randomized values.

\textbf{Institution classification.} To identify the communities
represented, we used the Carnegie classifications to group US-based
academic institutions into R1 research (very high research activity), R2
research (high research activity), four-year medical schools, or low
research (i.e., not R1, R2, or medical school) (27). Research institutes
(e.g., Mayo Clinic, Cold Springs Harbor), industry (e.g.,
pharmaceutical), and federal (e.g., NIH, CDC) research groups were
identified using the internet. Four-year medical schools and research
institutions were grouped together since these typically share research
prestige and have considerable resources to support research. Industry
and federal research were their own groups. The ``Other'' category
represents uncategorized US institutions. Non-US institutions were their
own category.

\textbf{Gender inference.} The genderize.io API was used to infer an
individual's gender based on their given name and country where
possible. The genderize.io platform uses data gathered from social media
to infer gender based on given names with the option to include an
associated language or country to enhance the probability of successful
inference. Since all manuscripts were submitted in English, which
precludes language association for names with special characters, names
were standardized to ASCII coding (e.g., ``José'' to ``Jose''). We next
matched each individuals' country against the list of 242 country names
accepted by genderize.io. Using the GenderGuesser package for R, all
unique given names associated with an accepted country were submitted to
the genderize.io API and any names returned without an inferred value of
either male or female were resubmitted without an associated country.
The data returned include the name, inferred gender (as ``male'',
``female'', ``unknown''), the probability of correct gender inference
(ranging from 0.5 to 1.0), and the number of instances the name and
gender were associated together (1 or greater). The inferred genders of
all given names (with and without an associated country) whose
probabilities were greater or equal to a modified probability (pmod) of
0.85 were used to infer genders (man/woman) of the individuals in our
data set (Supp Text). The presenting gender (man/woman) of editors and
senior editors in our data set was inferred by hand using Google where
possible, and the algorithm was validated using both editor and
published data (Supp Text)(5).

\textbf{Manuscript outcome analysis.} To better visualize and understand
the differences in outcomes according to author gender, we calculated
the difference in percentage points between the proportion of that
outcome for men and women. To correct for the disparity in the
participation of women relative to men at the ASM journals, all
percentage point comparisons were made relative to the gender and
population in question. For instance, the percentage point difference in
acceptance rates was the acceptance rate for men minus the acceptance
rate for women. A positive value indicated that men received the outcome
more often than women, whereas a negative value indicated that women
outperformed men in the given metric.

\textbf{Logistic regression models.} For the L2-regularized logistic
regression models, we established modeling pipelines for a binary
prediction task (65). First, we randomly split the data into training
and test sets so that the training set consisted of 80\% of the full
data set while the test set was composed of the remaining 20\% of the
data. To maintain the distribution of the two model outcomes found with
the full data set, we performed stratified splits. The training data was
used to build the models and the test set was used for evaluating
predictive performance. To build the models, we performed an internal
five-fold cross-validation where we tuned the cost hyper-parameter,
which determines the regularization strength where smaller values
specify stronger regularization. This internal cross-validation was
repeated 100 times. Then, we trained the full training data set with the
selected hyper-parameter values and applied the model to the held-out
data to evaluate the testing predictive performance of each model. The
data-split, hyper-parameter selection, training and testing steps were
repeated 25 times to get a reliable and robust reading of model
performance. Models were trained using the machine learning wrapper
caret package (v.6.0.81) in R (v.3.5.0).

\textbf{Code and data availability.} Data and code for all analysis
steps, logistic regression pipeline, and an Rmarkdown version of this
manuscript, are available at
\url{https://github.com/SchlossLab/Hagan_Gender_mBio_2019/}

\textbf{Acknowledgments} We would like to thank Nicole Broderick and
Arturo Casadevall for providing their data set for genderize validation
and acknowledge Arianna Miles-Jay and Joshua MA Stough for their
comments.

A.K.H. was responsible for data aggregation, analysis, interpretation,
and drafting the manuscript. B.T. completed the logistic regression
models. M.G. verified editor genders. A.K.H., H.B., and P.D.S. were
involved with conceptual development and revisions. Funding and
resources were provided by P.D.S. All authors contributed to the final
manuscript. P.D.S. is Chair of ASM Journals and A.K.H. was ASM staff
prior to publication of the analysis. B.T., M.G., and H.B. report no
conflict of interest.

Funding and access to the data for this work were provided by the
American Society for Microbiology. Early drafts were read by the ASM
Journals Committee with minimal influence on content or interpretation.

\newpage

\hypertarget{references}{%
\subsection{References}\label{references}}

\hypertarget{refs}{}
\leavevmode\hypertarget{ref-sheltzer_elite_2014}{}%
1. \textbf{Sheltzer JM}, \textbf{Smith JC}. 2014. Elite male faculty in
the life sciences employ fewer women. Proceedings of the National
Academy of Sciences \textbf{111}:10107--10112.
doi:\href{https://doi.org/10.1073/pnas.1403334111}{10.1073/pnas.1403334111}.

\leavevmode\hypertarget{ref-moss-racusin_science_2012}{}%
2. \textbf{Moss-Racusin CA}, \textbf{Dovidio JF}, \textbf{Brescoll VL},
\textbf{Graham MJ}, \textbf{Handelsman J}. 2012. Science faculty's
subtle gender biases favor male students. Proceedings of the National
Academy of Sciences \textbf{109}:16474--16479.
doi:\href{https://doi.org/10.1073/pnas.1211286109}{10.1073/pnas.1211286109}.

\leavevmode\hypertarget{ref-Ceci2012}{}%
3. \textbf{Ceci S}, \textbf{Williams W}. 2012. When scientists choose
motherhood. American Scientist \textbf{100}:138.
doi:\href{https://doi.org/10.1511/2012.95.138}{10.1511/2012.95.138}.

\leavevmode\hypertarget{ref-aakhus_gender_2018}{}%
4. \textbf{Aakhus E}, \textbf{Mitra N}, \textbf{Lautenbach E},
\textbf{Joffe S}. 2018. Gender and Byline Placement of Co-first Authors
in Clinical and Basic Science Journals With High Impact Factors. JAMA
\textbf{319}:610.
doi:\href{https://doi.org/10.1001/jama.2017.18672}{10.1001/jama.2017.18672}.

\leavevmode\hypertarget{ref-broderick_gender_2019}{}%
5. \textbf{Broderick NA}, \textbf{Casadevall A}. 2019. Gender
inequalities among authors who contributed equally. eLife
\textbf{8}:e36399.
doi:\href{https://doi.org/10.7554/eLife.36399}{10.7554/eLife.36399}.

\leavevmode\hypertarget{ref-BlairLoy2017}{}%
6. \textbf{Blair-Loy M}, \textbf{Rogers L}, \textbf{Glaser D},
\textbf{Wong Y}, \textbf{Abraham D}, \textbf{Cosman P}. 2017. Gender in
engineering departments: Are there gender differences in interruptions
of academic job talks? Social Sciences \textbf{6}:29.
doi:\href{https://doi.org/10.3390/socsci6010029}{10.3390/socsci6010029}.

\leavevmode\hypertarget{ref-symonds_gender_2006}{}%
7. \textbf{Symonds MRE}, \textbf{Gemmell NJ}, \textbf{Braisher TL},
\textbf{Gorringe KL}, \textbf{Elgar MA}. 2006. Gender Differences in
Publication Output: Towards an Unbiased Metric of Research Performance.
PLoS ONE \textbf{1}:e127.
doi:\href{https://doi.org/10.1371/journal.pone.0000127}{10.1371/journal.pone.0000127}.

\leavevmode\hypertarget{ref-Roper2019}{}%
8. \textbf{Roper RL}. 2019. Does gender bias still affect women in
science? Microbiology and Molecular Biology Reviews \textbf{83}.
doi:\href{https://doi.org/10.1128/mmbr.00018-19}{10.1128/mmbr.00018-19}.

\leavevmode\hypertarget{ref-diprete_cumulative_2006}{}%
9. \textbf{DiPrete TA}, \textbf{Eirich GM}. 2006. Cumulative Advantage
as a Mechanism for Inequality: A Review of Theoretical and Empirical
Developments. Annual Review of Sociology \textbf{32}:271--297.
doi:\href{https://doi.org/10.1146/annurev.soc.32.061604.123127}{10.1146/annurev.soc.32.061604.123127}.

\leavevmode\hypertarget{ref-thebaud_segregation_2018}{}%
10. \textbf{Thébaud S}, \textbf{Charles M}. 2018. Segregation,
Stereotypes, and STEM. Social Sciences \textbf{7}:111.
doi:\href{https://doi.org/10.3390/socsci7070111}{10.3390/socsci7070111}.

\leavevmode\hypertarget{ref-Iwasaki2020}{}%
11. \textbf{Iwasaki A}, \textbf{Monack DM}, \textbf{Cherry S},
\textbf{Harris NL}, \textbf{Subbarao K}, \textbf{Ramakrishnan
\textnormal{Lalita}}, \textbf{Pfeiffer JK}, \textbf{Cossart P},
\textbf{McFall-Ngai M}. 2020. Gender inclusion in microbial sciences.
Cell Host \& Microbe \textbf{27}:322--324.
doi:\href{https://doi.org/10.1016/j.chom.2020.02.013}{10.1016/j.chom.2020.02.013}.

\leavevmode\hypertarget{ref-Schloss2017}{}%
12. \textbf{Schloss PD}, \textbf{Johnston M}, \textbf{Casadevall A}.
2017. Support science by publishing in scientific society journals. mBio
\textbf{8}.
doi:\href{https://doi.org/10.1128/mbio.01633-17}{10.1128/mbio.01633-17}.

\leavevmode\hypertarget{ref-lerback_journals_2017}{}%
13. \textbf{Lerback J}, \textbf{Hanson B}. 2017. Journals invite too few
women to referee. Nature \textbf{541}:455--457.
doi:\href{https://doi.org/10.1038/541455a}{10.1038/541455a}.

\leavevmode\hypertarget{ref-fox_editor_2016}{}%
14. \textbf{Fox CW}, \textbf{Burns CS}, \textbf{Meyer JA}. 2016. Editor
and reviewer gender influence the peer review process but not peer
review outcomes at an ecology journal. Functional Ecology
\textbf{30}:140--153.
doi:\href{https://doi.org/10.1111/1365-2435.12529}{10.1111/1365-2435.12529}.

\leavevmode\hypertarget{ref-ceci_understanding_2011}{}%
15. \textbf{Ceci SJ}, \textbf{Williams WM}. 2011. Understanding current
causes of women's underrepresentation in science. Proceedings of the
National Academy of Sciences \textbf{108}:3157--3162.
doi:\href{https://doi.org/10.1073/pnas.1014871108}{10.1073/pnas.1014871108}.

\leavevmode\hypertarget{ref-handley_examination_2015}{}%
16. \textbf{Handley G}, \textbf{Frantz CM}, \textbf{Kocovsky PM},
\textbf{DeVries DR}, \textbf{Cooke SJ}, \textbf{Claussen J}. 2015. An
Examination of Gender Differences in the American Fisheries Society
Peer-Review Process. Fisheries \textbf{40}:442--451.
doi:\href{https://doi.org/10.1080/03632415.2015.1059824}{10.1080/03632415.2015.1059824}.

\leavevmode\hypertarget{ref-edwards_gender_2018}{}%
17. \textbf{Edwards HA}, \textbf{Schroeder J}, \textbf{Dugdale HL}.
2018. Gender differences in authorships are not associated with
publication bias in an evolutionary journal. PLOS ONE
\textbf{13}:e0201725.
doi:\href{https://doi.org/10.1371/journal.pone.0201725}{10.1371/journal.pone.0201725}.

\leavevmode\hypertarget{ref-buckley_is_2014}{}%
18. \textbf{Buckley HL}, \textbf{Sciligo AR}, \textbf{Adair KL},
\textbf{Case BS}, \textbf{Monks JM}. 2014. Is there gender bias in
reviewer selection and publication success rates for the. New Zealand
Journal of Ecology \textbf{38}:5.

\leavevmode\hypertarget{ref-Whelan2019}{}%
19. \textbf{Whelan AM}, \textbf{Schimel DS}. 2019. Authorship and gender
in ESA journals. The Bulletin of the Ecological Society of America
\textbf{100}.
doi:\href{https://doi.org/10.1002/bes2.1567}{10.1002/bes2.1567}.

\leavevmode\hypertarget{ref-Murray400515}{}%
20. \textbf{Murray D}, \textbf{Siler K}, \textbf{Larivière V},
\textbf{Chan WM}, \textbf{Collings AM}, \textbf{Raymond J},
\textbf{Sugimoto CR}. 2019. Author-reviewer homophily in peer review.
bioRxiv. doi:\href{https://doi.org/10.1101/400515}{10.1101/400515}.

\leavevmode\hypertarget{ref-fox_gender_2019}{}%
21. \textbf{Fox CW}, \textbf{Paine CET}. 2019. Gender differences in
peer review outcomes and manuscript impact at six journals of ecology
and evolution. Ecology and Evolution \textbf{9}:3599--3619.
doi:\href{https://doi.org/10.1002/ece3.4993}{10.1002/ece3.4993}.

\leavevmode\hypertarget{ref-Physics_2018}{}%
22. 2018. Diversity and inclusion in peer review at IOP publishing. IOP
Publishing.
\url{http://www.ioppublishing.org/wp-content/uploads/2018/09/J-VAR-BK-0818-PRW-report-final2.pdf}.

\leavevmode\hypertarget{ref-RoyalChem_2019}{}%
23. 2019. Is publishing in the chemical sciences gender biased? Royal
Society of Chemistry.
\url{https://www.rsc.org/globalassets/04-campaigning-outreach/campaigning/gender-bias/gender-bias-report-final.pdf}.

\leavevmode\hypertarget{ref-chang_2020}{}%
24. \textbf{Chang A}. 2020. Building an equitable and inclusive culture
at ASM. ASM.org.
\url{https://asm.org/Articles/2020/April/ASM-Guiding-Principles-on-Diversity,-Equity-and-In}.

\leavevmode\hypertarget{ref-Caputo2018}{}%
25. \textbf{Caputo RK}. 2018. Peer review: A vital gatekeeping function
and obligation of professional scholarly practice. Families in Society:
The Journal of Contemporary Social Services \textbf{100}:6--16.
doi:\href{https://doi.org/10.1177/1044389418808155}{10.1177/1044389418808155}.

\leavevmode\hypertarget{ref-Chowdhry2015}{}%
26. \textbf{Chowdhry A}. 2015. Gatekeepers of the academic world: A
recipe for good peer review. Advances in Medical Education and Practice
329.
doi:\href{https://doi.org/10.2147/amep.s83887}{10.2147/amep.s83887}.

\leavevmode\hypertarget{ref-Carnegie2018}{}%
27. 2018. Carnegie classification of institutions of higher education.
Indiana University Center for Postsecondary Research.
\url{http://carnegieclassifications.iu.edu}.

\leavevmode\hypertarget{ref-allagnat_gender_2017}{}%
28. \textbf{Allagnat L}, \textbf{Berghmans S}, \textbf{Falk-Krzesinski
HJ}, \textbf{Hanafi S}, \textbf{Herbert R}, \textbf{Huggett S},
\textbf{Tobin S}. 2017. Gender in the global research landscape 96.
doi:\href{https://doi.org/10.17632/bb3cjfgm2w.2}{10.17632/bb3cjfgm2w.2}.

\leavevmode\hypertarget{ref-erin_hengel_publishing_2017}{}%
29. \textbf{Hengel E}. 2017. Publishing while female 1--64.
doi:\href{https://doi.org/10.17863/CAM.17548}{10.17863/CAM.17548}.

\leavevmode\hypertarget{ref-weeden_degrees_2017}{}%
30. \textbf{Weeden K}, \textbf{Thébaud S}, \textbf{Gelbgiser D}. 2017.
Degrees of Difference: Gender Segregation of U.S. Doctorates by Field
and Program Prestige. Sociological Science \textbf{4}:123--150.
doi:\href{https://doi.org/10.15195/v4.a6}{10.15195/v4.a6}.

\leavevmode\hypertarget{ref-Dotson2012}{}%
31. \textbf{Dotson K}. 2012. HOW IS THIS PAPER PHILOSOPHY? Comparative
Philosophy: An International Journal of Constructive Engagement of
Distinct Approaches toward World Philosophy \textbf{3}.
doi:\href{https://doi.org/10.31979/2151-6014(2012).030105}{10.31979/2151-6014(2012).030105}.

\leavevmode\hypertarget{ref-Dotson2014}{}%
32. \textbf{Dotson K}. 2014. Conceptualizing epistemic oppression.
Social Epistemology \textbf{28}:115--138.
doi:\href{https://doi.org/10.1080/02691728.2013.782585}{10.1080/02691728.2013.782585}.

\leavevmode\hypertarget{ref-PrescodWeinstein2020}{}%
33. \textbf{Prescod-Weinstein C}. 2020. Making black women scientists
under white empiricism: The racialization of epistemology in physics.
Signs: Journal of Women in Culture and Society \textbf{45}:421--447.
doi:\href{https://doi.org/10.1086/704991}{10.1086/704991}.

\leavevmode\hypertarget{ref-Hoppe2019}{}%
34. \textbf{Hoppe TA}, \textbf{Litovitz A}, \textbf{Willis KA},
\textbf{Meseroll RA}, \textbf{Perkins MJ}, \textbf{Hutchins BI},
\textbf{Davis AF}, \textbf{Lauer MS}, \textbf{Valantine HA},
\textbf{Anderson JM}, \textbf{Santangelo GM}. 2019. Topic choice
contributes to the lower rate of NIH awards to African-American/Black
scientists. Science Advances \textbf{5}:eaaw7238.
doi:\href{https://doi.org/10.1126/sciadv.aaw7238}{10.1126/sciadv.aaw7238}.

\leavevmode\hypertarget{ref-debarre_gender_2018}{}%
35. \textbf{Débarre F}, \textbf{Rode NO}, \textbf{Ugelvig LV}. 2018.
Gender equity at scientific events. Evolution Letters
\textbf{2}:148--158.
doi:\href{https://doi.org/10.1002/evl3.49}{10.1002/evl3.49}.

\leavevmode\hypertarget{ref-sardelis_not_2016}{}%
36. \textbf{Sardelis S}, \textbf{Drew JA}. 2016. Not ``Pulling up the
Ladder'': Women Who Organize Conference Symposia Provide Greater
Opportunities for Women to Speak at Conservation Conferences. PLOS ONE
\textbf{11}:e0160015.
doi:\href{https://doi.org/10.1371/journal.pone.0160015}{10.1371/journal.pone.0160015}.

\leavevmode\hypertarget{ref-casadevall_presence_2014}{}%
37. \textbf{Casadevall A}, \textbf{Handelsman J}. 2014. The Presence of
Female Conveners Correlates with a Higher Proportion of Female Speakers
at Scientific Symposia. mBio \textbf{5}:e00846--13--e00846--13.
doi:\href{https://doi.org/10.1128/mBio.00846-13}{10.1128/mBio.00846-13}.

\leavevmode\hypertarget{ref-guarino_faculty_2017}{}%
38. \textbf{Guarino CM}, \textbf{Borden VMH}. 2017. Faculty Service
Loads and Gender: Are Women Taking Care of the Academic Family? Research
in Higher Education \textbf{58}:672--694.
doi:\href{https://doi.org/10.1007/s11162-017-9454-2}{10.1007/s11162-017-9454-2}.

\leavevmode\hypertarget{ref-wiedman_rewarding_2019}{}%
39. \textbf{Wiedman C}. 2019. Rewarding Collaborative Research: Role
Congruity Bias and the Gender Pay Gap in Academe. Journal of Business
Ethics.
doi:\href{https://doi.org/10.1007/s10551-019-04165-0}{10.1007/s10551-019-04165-0}.

\leavevmode\hypertarget{ref-fox_citations_2016}{}%
40. \textbf{Fox CW}, \textbf{Paine CET}, \textbf{Sauterey B}. 2016.
Citations increase with manuscript length, author number, and references
cited in ecology journals. Ecology and Evolution \textbf{6}:7717--7726.
doi:\href{https://doi.org/10.1002/ece3.2505}{10.1002/ece3.2505}.

\leavevmode\hypertarget{ref-holman_researchers_2019}{}%
41. \textbf{Holman L}, \textbf{Morandin C}. 2019. Researchers
collaborate with same-gendered colleagues more often than expected
across the life sciences. PLOS ONE \textbf{14}:e0216128.
doi:\href{https://doi.org/10.1371/journal.pone.0216128}{10.1371/journal.pone.0216128}.

\leavevmode\hypertarget{ref-Salerno2019}{}%
42. \textbf{Salerno PE}, \textbf{Páez-Vacas M}, \textbf{Guayasamin JM},
\textbf{Stynoski JL}. 2019. Male principal investigators (almost) don't
publish with women in ecology and zoology. PLOS ONE
\textbf{14}:e0218598.
doi:\href{https://doi.org/10.1371/journal.pone.0218598}{10.1371/journal.pone.0218598}.

\leavevmode\hypertarget{ref-Fox2018}{}%
43. \textbf{Fox CW}, \textbf{Ritchey JP}, \textbf{Paine CET}. 2018.
Patterns of authorship in ecology and evolution: First, last, and
corresponding authorship vary with gender and geography. Ecology and
Evolution \textbf{8}:11492--11507.
doi:\href{https://doi.org/10.1002/ece3.4584}{10.1002/ece3.4584}.

\leavevmode\hypertarget{ref-Fox2019}{}%
44. \textbf{Fox CW}, \textbf{Duffy MA}, \textbf{Fairbairn DJ},
\textbf{Meyer JA}. 2019. Gender diversity of editorial boards and gender
differences in the peer review process at six journals of ecology and
evolution. Ecology and Evolution \textbf{9}:13636--13649.
doi:\href{https://doi.org/10.1002/ece3.5794}{10.1002/ece3.5794}.

\leavevmode\hypertarget{ref-Kolev2019}{}%
45. \textbf{Kolev J}, \textbf{Fuentes-Medel Y}, \textbf{Murray F}. 2019.
Is blinded review enough? How gendered outcomes arise even under
anonymous evaluation.
doi:\href{https://doi.org/10.3386/w25759}{10.3386/w25759}.

\leavevmode\hypertarget{ref-Brown2012}{}%
46. \textbf{Brown GD}, \textbf{Denning DW}, \textbf{Gow NAR},
\textbf{Levitz SM}, \textbf{Netea MG}, \textbf{White TC}. 2012. Hidden
killers: Human fungal infections. Science Translational Medicine
\textbf{4}:165rv13--165rv13.
doi:\href{https://doi.org/10.1126/scitranslmed.3004404}{10.1126/scitranslmed.3004404}.

\leavevmode\hypertarget{ref-Bongomin2017}{}%
47. \textbf{Bongomin F}, \textbf{Gago S}, \textbf{Oladele R},
\textbf{Denning D}. 2017. Global and multi-national prevalence of fungal
diseasesEstimate precision. Journal of Fungi \textbf{3}:57.
doi:\href{https://doi.org/10.3390/jof3040057}{10.3390/jof3040057}.

\leavevmode\hypertarget{ref-Vallabhaneni2016}{}%
48. \textbf{Vallabhaneni S}, \textbf{Mody RK}, \textbf{Walker T},
\textbf{Chiller T}. 2016. The global burden of fungal diseases.
Infectious Disease Clinics of North America \textbf{30}:1--11.
doi:\href{https://doi.org/10.1016/j.idc.2015.10.004}{10.1016/j.idc.2015.10.004}.

\leavevmode\hypertarget{ref-ASM_2019}{}%
49. \textbf{Konopka JB}, \textbf{Casadevall A}, \textbf{Taylor JW},
\textbf{Heitman J}, \textbf{Cowen L}. One health: Fungal pathogens of
humans, animals, and plants. American Academy of Microbiology.
\url{https://www.asmscience.org/content/colloquia.56}.

\leavevmode\hypertarget{ref-potvin_diversity_2018}{}%
50. \textbf{Potvin DA}, \textbf{Burdfield-Steel E}, \textbf{Potvin JM},
\textbf{Heap SM}. 2018. Diversity begets diversity: A global perspective
on gender equality in scientific society leadership. PLOS ONE
\textbf{13}:e0197280.
doi:\href{https://doi.org/10.1371/journal.pone.0197280}{10.1371/journal.pone.0197280}.

\leavevmode\hypertarget{ref-Schloss2020}{}%
51. \textbf{Schloss PD}, \textbf{Junior M}, \textbf{Alvania R},
\textbf{Arias CA}, \textbf{Baumler A}, \textbf{Casadevall A},
\textbf{Detweiler C}, \textbf{Drake H}, \textbf{Gilbert J},
\textbf{Imperiale MJ}, \textbf{Lovett S}, \textbf{Maloy S},
\textbf{McAdam AJ}, \textbf{Newton ILG}, \textbf{Sadowsky MJ},
\textbf{Sandri-Goldin RM}, \textbf{Silhavy TJ}, \textbf{Tontonoz P},
\textbf{Young J-AH}, \textbf{Cameron CE}, \textbf{Cann I},
\textbf{Fuller AO}, \textbf{Kozik AJ}. 2020. The ASM journals committee
values the contributions of black microbiologists. mBio \textbf{11}.
doi:\href{https://doi.org/10.1128/mbio.01998-20}{10.1128/mbio.01998-20}.

\leavevmode\hypertarget{ref-Hagan2020}{}%
52. \textbf{Hagan AK}, \textbf{Pollet RM}, \textbf{Libertucci J}. 2020.
Suggestions for improving invited speaker diversity to reflect trainee
diversity. Journal of Microbiology \& Biology Education \textbf{21}.
doi:\href{https://doi.org/10.1128/jmbe.v21i1.2105}{10.1128/jmbe.v21i1.2105}.

\leavevmode\hypertarget{ref-fox_gender_2016}{}%
53. \textbf{Fox CW}, \textbf{Burns CS}, \textbf{Muncy AD}, \textbf{Meyer
JA}. 2016. Gender differences in patterns of authorship do not affect
peer review outcomes at an ecology journal. Functional Ecology
\textbf{30}:126--139.
doi:\href{https://doi.org/10.1111/1365-2435.12587}{10.1111/1365-2435.12587}.

\leavevmode\hypertarget{ref-cox_case_2018}{}%
54. \textbf{Cox AR}, \textbf{Montgomerie R}. 2018. The Case For and
Against Double-blind Reviews. BioRxiv.
doi:\href{https://doi.org/10.1101/495465}{10.1101/495465}.

\leavevmode\hypertarget{ref-applebaum_remediating_2019}{}%
55. \textbf{Applebaum B}. 2019. Remediating Campus Climate: Implicit
Bias Training is Not Enough. Studies in Philosophy and Education
\textbf{38}:129--141.
doi:\href{https://doi.org/10.1007/s11217-018-9644-1}{10.1007/s11217-018-9644-1}.

\leavevmode\hypertarget{ref-Iverson2007}{}%
56. \textbf{Iverson SV}. 2007. Camouflaging power and privilege: A
critical race analysis of university diversity policies. Educational
Administration Quarterly \textbf{43}:586--611.
doi:\href{https://doi.org/10.1177/0013161x07307794}{10.1177/0013161x07307794}.

\leavevmode\hypertarget{ref-Tate2016}{}%
57. \textbf{Tate SA}, \textbf{Bagguley P}. 2016. Building the
anti-racist university: Next steps. Race Ethnicity and Education
\textbf{20}:289--299.
doi:\href{https://doi.org/10.1080/13613324.2016.1260227}{10.1080/13613324.2016.1260227}.

\leavevmode\hypertarget{ref-harvey_diversity_2011}{}%
58. \textbf{Harvey WB}. 2011. Higher education and diversity: Ethical
and practical responsiblity in the academy.
\url{http://www.kirwaninstitute.osu.edu/reports/2011/11_2011_HigherEducationandDiversity.pdf}.

\leavevmode\hypertarget{ref-Holmes2011}{}%
59. \textbf{Holmes MA}, \textbf{Asher P}, \textbf{Farrington J},
\textbf{Fine R}, \textbf{Leinen MS}, \textbf{LeBoy P}. 2011. Does gender
bias influence awards given by societies? Eos, Transactions American
Geophysical Union \textbf{92}:421--422.
doi:\href{https://doi.org/10.1029/2011eo470002}{10.1029/2011eo470002}.

\leavevmode\hypertarget{ref-Malouff2016}{}%
60. \textbf{Malouff JM}, \textbf{Thorsteinsson EB}. 2016. Bias in
grading: A meta-analysis of experimental research findings. Australian
Journal of Education \textbf{60}:245--256.
doi:\href{https://doi.org/10.1177/0004944116664618}{10.1177/0004944116664618}.

\leavevmode\hypertarget{ref-Reddy2010}{}%
61. \textbf{Reddy YM}, \textbf{Andrade H}. 2010. A review of rubric use
in higher education. Assessment \& Evaluation in Higher Education
\textbf{35}:435--448.
doi:\href{https://doi.org/10.1080/02602930902862859}{10.1080/02602930902862859}.

\leavevmode\hypertarget{ref-Rezaei2010}{}%
62. \textbf{Rezaei AR}, \textbf{Lovorn M}. 2010. Reliability and
validity of rubrics for assessment through writing. Assessing Writing
\textbf{15}:18--39.
doi:\href{https://doi.org/10.1016/j.asw.2010.01.003}{10.1016/j.asw.2010.01.003}.

\leavevmode\hypertarget{ref-Casadevall2014}{}%
63. \textbf{Casadevall A}, \textbf{Fang FC}. 2014. Causes for the
persistence of impact factor mania. mBio \textbf{5}.
doi:\href{https://doi.org/10.1128/mbio.00064-14}{10.1128/mbio.00064-14}.

\leavevmode\hypertarget{ref-Casadevall2015}{}%
64. \textbf{Casadevall A}, \textbf{Fang FC}. 2015. Impacted science:
Impact is not importance. mBio \textbf{6}.
doi:\href{https://doi.org/10.1128/mbio.01593-15}{10.1128/mbio.01593-15}.

\leavevmode\hypertarget{ref-Topuolu2020}{}%
65. \textbf{Topçuoğlu BD}, \textbf{Lesniak NA}, \textbf{Ruffin MT},
\textbf{Wiens J}, \textbf{Schloss PD}. 2020. A framework for effective
application of machine learning to microbiome-based classification
problems. mBio \textbf{11}.
doi:\href{https://doi.org/10.1128/mbio.00434-20}{10.1128/mbio.00434-20}.

\newpage

Table 1. Analysis of sub-discipline participation by women corresponding
authors at five ASM journals.

\begin{longtable}[]{@{}llrrrr@{}}
\toprule
\begin{minipage}[b]{0.06\columnwidth}\raggedright
Journal\strut
\end{minipage} & \begin{minipage}[b]{0.45\columnwidth}\raggedright
Category\strut
\end{minipage} & \begin{minipage}[b]{0.03\columnwidth}\raggedleft
N\strut
\end{minipage} & \begin{minipage}[b]{0.08\columnwidth}\raggedleft
\% Accepted\strut
\end{minipage} & \begin{minipage}[b]{0.11\columnwidth}\raggedleft
\% Women Editors\strut
\end{minipage} & \begin{minipage}[b]{0.11\columnwidth}\raggedleft
\% Women Authors\strut
\end{minipage}\tabularnewline
\midrule
\endhead
\begin{minipage}[t]{0.06\columnwidth}\raggedright
AAC\strut
\end{minipage} & \begin{minipage}[t]{0.45\columnwidth}\raggedright
Analytical Procedures\strut
\end{minipage} & \begin{minipage}[t]{0.03\columnwidth}\raggedleft
135\strut
\end{minipage} & \begin{minipage}[t]{0.08\columnwidth}\raggedleft
43.0\strut
\end{minipage} & \begin{minipage}[t]{0.11\columnwidth}\raggedleft
14\strut
\end{minipage} & \begin{minipage}[t]{0.11\columnwidth}\raggedleft
29\strut
\end{minipage}\tabularnewline
\begin{minipage}[t]{0.06\columnwidth}\raggedright
AAC\strut
\end{minipage} & \begin{minipage}[t]{0.45\columnwidth}\raggedright
Antiviral Agents\strut
\end{minipage} & \begin{minipage}[t]{0.03\columnwidth}\raggedleft
836\strut
\end{minipage} & \begin{minipage}[t]{0.08\columnwidth}\raggedleft
56.5\strut
\end{minipage} & \begin{minipage}[t]{0.11\columnwidth}\raggedleft
6\strut
\end{minipage} & \begin{minipage}[t]{0.11\columnwidth}\raggedleft
33\strut
\end{minipage}\tabularnewline
\begin{minipage}[t]{0.06\columnwidth}\raggedright
AAC\strut
\end{minipage} & \begin{minipage}[t]{0.45\columnwidth}\raggedright
Biologic Response Modifiers\strut
\end{minipage} & \begin{minipage}[t]{0.03\columnwidth}\raggedleft
44\strut
\end{minipage} & \begin{minipage}[t]{0.08\columnwidth}\raggedleft
40.9\strut
\end{minipage} & \begin{minipage}[t]{0.11\columnwidth}\raggedleft
12\strut
\end{minipage} & \begin{minipage}[t]{0.11\columnwidth}\raggedleft
43\strut
\end{minipage}\tabularnewline
\begin{minipage}[t]{0.06\columnwidth}\raggedright
AAC\strut
\end{minipage} & \begin{minipage}[t]{0.45\columnwidth}\raggedright
Chemistry; Biosynthesis\strut
\end{minipage} & \begin{minipage}[t]{0.03\columnwidth}\raggedleft
109\strut
\end{minipage} & \begin{minipage}[t]{0.08\columnwidth}\raggedleft
29.4\strut
\end{minipage} & \begin{minipage}[t]{0.11\columnwidth}\raggedleft
10\strut
\end{minipage} & \begin{minipage}[t]{0.11\columnwidth}\raggedleft
32\strut
\end{minipage}\tabularnewline
\begin{minipage}[t]{0.06\columnwidth}\raggedright
AAC\strut
\end{minipage} & \begin{minipage}[t]{0.45\columnwidth}\raggedright
Clinical Therapeutics\strut
\end{minipage} & \begin{minipage}[t]{0.03\columnwidth}\raggedleft
1060\strut
\end{minipage} & \begin{minipage}[t]{0.08\columnwidth}\raggedleft
48.9\strut
\end{minipage} & \begin{minipage}[t]{0.11\columnwidth}\raggedleft
13\strut
\end{minipage} & \begin{minipage}[t]{0.11\columnwidth}\raggedleft
31\strut
\end{minipage}\tabularnewline
\begin{minipage}[t]{0.06\columnwidth}\raggedright
AAC\strut
\end{minipage} & \begin{minipage}[t]{0.45\columnwidth}\raggedright
Epidemiology and Surveillance\strut
\end{minipage} & \begin{minipage}[t]{0.03\columnwidth}\raggedleft
765\strut
\end{minipage} & \begin{minipage}[t]{0.08\columnwidth}\raggedleft
52.3\strut
\end{minipage} & \begin{minipage}[t]{0.11\columnwidth}\raggedleft
14\strut
\end{minipage} & \begin{minipage}[t]{0.11\columnwidth}\raggedleft
40\strut
\end{minipage}\tabularnewline
\begin{minipage}[t]{0.06\columnwidth}\raggedright
AAC\strut
\end{minipage} & \begin{minipage}[t]{0.45\columnwidth}\raggedright
Experimental Therapeutics\strut
\end{minipage} & \begin{minipage}[t]{0.03\columnwidth}\raggedleft
1329\strut
\end{minipage} & \begin{minipage}[t]{0.08\columnwidth}\raggedleft
57.4\strut
\end{minipage} & \begin{minipage}[t]{0.11\columnwidth}\raggedleft
13\strut
\end{minipage} & \begin{minipage}[t]{0.11\columnwidth}\raggedleft
28\strut
\end{minipage}\tabularnewline
\begin{minipage}[t]{0.06\columnwidth}\raggedright
AAC\strut
\end{minipage} & \begin{minipage}[t]{0.45\columnwidth}\raggedright
FDA Approvals\strut
\end{minipage} & \begin{minipage}[t]{0.03\columnwidth}\raggedleft
1\strut
\end{minipage} & \begin{minipage}[t]{0.08\columnwidth}\raggedleft
NA\strut
\end{minipage} & \begin{minipage}[t]{0.11\columnwidth}\raggedleft
NA\strut
\end{minipage} & \begin{minipage}[t]{0.11\columnwidth}\raggedleft
NA\strut
\end{minipage}\tabularnewline
\begin{minipage}[t]{0.06\columnwidth}\raggedright
AAC\strut
\end{minipage} & \begin{minipage}[t]{0.45\columnwidth}\raggedright
Mechanisms of Action: Physiological Effects\strut
\end{minipage} & \begin{minipage}[t]{0.03\columnwidth}\raggedleft
597\strut
\end{minipage} & \begin{minipage}[t]{0.08\columnwidth}\raggedleft
51.8\strut
\end{minipage} & \begin{minipage}[t]{0.11\columnwidth}\raggedleft
14\strut
\end{minipage} & \begin{minipage}[t]{0.11\columnwidth}\raggedleft
30\strut
\end{minipage}\tabularnewline
\begin{minipage}[t]{0.06\columnwidth}\raggedright
AAC\strut
\end{minipage} & \begin{minipage}[t]{0.45\columnwidth}\raggedright
Mechanisms of Resistance\strut
\end{minipage} & \begin{minipage}[t]{0.03\columnwidth}\raggedleft
1783\strut
\end{minipage} & \begin{minipage}[t]{0.08\columnwidth}\raggedleft
60.0\strut
\end{minipage} & \begin{minipage}[t]{0.11\columnwidth}\raggedleft
14\strut
\end{minipage} & \begin{minipage}[t]{0.11\columnwidth}\raggedleft
36\strut
\end{minipage}\tabularnewline
\begin{minipage}[t]{0.06\columnwidth}\raggedright
AAC\strut
\end{minipage} & \begin{minipage}[t]{0.45\columnwidth}\raggedright
Pharmacology\strut
\end{minipage} & \begin{minipage}[t]{0.03\columnwidth}\raggedleft
878\strut
\end{minipage} & \begin{minipage}[t]{0.08\columnwidth}\raggedleft
66.6\strut
\end{minipage} & \begin{minipage}[t]{0.11\columnwidth}\raggedleft
13\strut
\end{minipage} & \begin{minipage}[t]{0.11\columnwidth}\raggedleft
29\strut
\end{minipage}\tabularnewline
\begin{minipage}[t]{0.06\columnwidth}\raggedright
AAC\strut
\end{minipage} & \begin{minipage}[t]{0.45\columnwidth}\raggedright
Susceptibility\strut
\end{minipage} & \begin{minipage}[t]{0.03\columnwidth}\raggedleft
1051\strut
\end{minipage} & \begin{minipage}[t]{0.08\columnwidth}\raggedleft
46.8\strut
\end{minipage} & \begin{minipage}[t]{0.11\columnwidth}\raggedleft
12\strut
\end{minipage} & \begin{minipage}[t]{0.11\columnwidth}\raggedleft
39\strut
\end{minipage}\tabularnewline
\begin{minipage}[t]{0.06\columnwidth}\raggedright
AEM\strut
\end{minipage} & \begin{minipage}[t]{0.45\columnwidth}\raggedright
Biodegradation\strut
\end{minipage} & \begin{minipage}[t]{0.03\columnwidth}\raggedleft
302\strut
\end{minipage} & \begin{minipage}[t]{0.08\columnwidth}\raggedleft
38.4\strut
\end{minipage} & \begin{minipage}[t]{0.11\columnwidth}\raggedleft
35\strut
\end{minipage} & \begin{minipage}[t]{0.11\columnwidth}\raggedleft
26\strut
\end{minipage}\tabularnewline
\begin{minipage}[t]{0.06\columnwidth}\raggedright
AEM\strut
\end{minipage} & \begin{minipage}[t]{0.45\columnwidth}\raggedright
Biotechnology\strut
\end{minipage} & \begin{minipage}[t]{0.03\columnwidth}\raggedleft
802\strut
\end{minipage} & \begin{minipage}[t]{0.08\columnwidth}\raggedleft
37.9\strut
\end{minipage} & \begin{minipage}[t]{0.11\columnwidth}\raggedleft
30\strut
\end{minipage} & \begin{minipage}[t]{0.11\columnwidth}\raggedleft
27\strut
\end{minipage}\tabularnewline
\begin{minipage}[t]{0.06\columnwidth}\raggedright
AEM\strut
\end{minipage} & \begin{minipage}[t]{0.45\columnwidth}\raggedright
Environmental Microbiology\strut
\end{minipage} & \begin{minipage}[t]{0.03\columnwidth}\raggedleft
2395\strut
\end{minipage} & \begin{minipage}[t]{0.08\columnwidth}\raggedleft
30.3\strut
\end{minipage} & \begin{minipage}[t]{0.11\columnwidth}\raggedleft
35\strut
\end{minipage} & \begin{minipage}[t]{0.11\columnwidth}\raggedleft
42\strut
\end{minipage}\tabularnewline
\begin{minipage}[t]{0.06\columnwidth}\raggedright
AEM\strut
\end{minipage} & \begin{minipage}[t]{0.45\columnwidth}\raggedright
Enzymology and Protein Engineering\strut
\end{minipage} & \begin{minipage}[t]{0.03\columnwidth}\raggedleft
340\strut
\end{minipage} & \begin{minipage}[t]{0.08\columnwidth}\raggedleft
46.5\strut
\end{minipage} & \begin{minipage}[t]{0.11\columnwidth}\raggedleft
28\strut
\end{minipage} & \begin{minipage}[t]{0.11\columnwidth}\raggedleft
24\strut
\end{minipage}\tabularnewline
\begin{minipage}[t]{0.06\columnwidth}\raggedright
AEM\strut
\end{minipage} & \begin{minipage}[t]{0.45\columnwidth}\raggedright
Evolutionary and Genomic Microbiology\strut
\end{minipage} & \begin{minipage}[t]{0.03\columnwidth}\raggedleft
279\strut
\end{minipage} & \begin{minipage}[t]{0.08\columnwidth}\raggedleft
48.4\strut
\end{minipage} & \begin{minipage}[t]{0.11\columnwidth}\raggedleft
32\strut
\end{minipage} & \begin{minipage}[t]{0.11\columnwidth}\raggedleft
30\strut
\end{minipage}\tabularnewline
\begin{minipage}[t]{0.06\columnwidth}\raggedright
AEM\strut
\end{minipage} & \begin{minipage}[t]{0.45\columnwidth}\raggedright
Food Microbiology\strut
\end{minipage} & \begin{minipage}[t]{0.03\columnwidth}\raggedleft
1216\strut
\end{minipage} & \begin{minipage}[t]{0.08\columnwidth}\raggedleft
38.2\strut
\end{minipage} & \begin{minipage}[t]{0.11\columnwidth}\raggedleft
33\strut
\end{minipage} & \begin{minipage}[t]{0.11\columnwidth}\raggedleft
39\strut
\end{minipage}\tabularnewline
\begin{minipage}[t]{0.06\columnwidth}\raggedright
AEM\strut
\end{minipage} & \begin{minipage}[t]{0.45\columnwidth}\raggedright
Genetics and Molecular Biology\strut
\end{minipage} & \begin{minipage}[t]{0.03\columnwidth}\raggedleft
587\strut
\end{minipage} & \begin{minipage}[t]{0.08\columnwidth}\raggedleft
51.8\strut
\end{minipage} & \begin{minipage}[t]{0.11\columnwidth}\raggedleft
32\strut
\end{minipage} & \begin{minipage}[t]{0.11\columnwidth}\raggedleft
36\strut
\end{minipage}\tabularnewline
\begin{minipage}[t]{0.06\columnwidth}\raggedright
AEM\strut
\end{minipage} & \begin{minipage}[t]{0.45\columnwidth}\raggedright
Geomicrobiology\strut
\end{minipage} & \begin{minipage}[t]{0.03\columnwidth}\raggedleft
151\strut
\end{minipage} & \begin{minipage}[t]{0.08\columnwidth}\raggedleft
44.4\strut
\end{minipage} & \begin{minipage}[t]{0.11\columnwidth}\raggedleft
34\strut
\end{minipage} & \begin{minipage}[t]{0.11\columnwidth}\raggedleft
37\strut
\end{minipage}\tabularnewline
\begin{minipage}[t]{0.06\columnwidth}\raggedright
AEM\strut
\end{minipage} & \begin{minipage}[t]{0.45\columnwidth}\raggedright
Invertebrate Microbiology\strut
\end{minipage} & \begin{minipage}[t]{0.03\columnwidth}\raggedleft
317\strut
\end{minipage} & \begin{minipage}[t]{0.08\columnwidth}\raggedleft
45.7\strut
\end{minipage} & \begin{minipage}[t]{0.11\columnwidth}\raggedleft
29\strut
\end{minipage} & \begin{minipage}[t]{0.11\columnwidth}\raggedleft
37\strut
\end{minipage}\tabularnewline
\begin{minipage}[t]{0.06\columnwidth}\raggedright
AEM\strut
\end{minipage} & \begin{minipage}[t]{0.45\columnwidth}\raggedright
Methods\strut
\end{minipage} & \begin{minipage}[t]{0.03\columnwidth}\raggedleft
529\strut
\end{minipage} & \begin{minipage}[t]{0.08\columnwidth}\raggedleft
39.7\strut
\end{minipage} & \begin{minipage}[t]{0.11\columnwidth}\raggedleft
30\strut
\end{minipage} & \begin{minipage}[t]{0.11\columnwidth}\raggedleft
29\strut
\end{minipage}\tabularnewline
\begin{minipage}[t]{0.06\columnwidth}\raggedright
AEM\strut
\end{minipage} & \begin{minipage}[t]{0.45\columnwidth}\raggedright
Microbial Ecology\strut
\end{minipage} & \begin{minipage}[t]{0.03\columnwidth}\raggedleft
1121\strut
\end{minipage} & \begin{minipage}[t]{0.08\columnwidth}\raggedleft
35.8\strut
\end{minipage} & \begin{minipage}[t]{0.11\columnwidth}\raggedleft
29\strut
\end{minipage} & \begin{minipage}[t]{0.11\columnwidth}\raggedleft
37\strut
\end{minipage}\tabularnewline
\begin{minipage}[t]{0.06\columnwidth}\raggedright
AEM\strut
\end{minipage} & \begin{minipage}[t]{0.45\columnwidth}\raggedright
Mycology\strut
\end{minipage} & \begin{minipage}[t]{0.03\columnwidth}\raggedleft
73\strut
\end{minipage} & \begin{minipage}[t]{0.08\columnwidth}\raggedleft
47.9\strut
\end{minipage} & \begin{minipage}[t]{0.11\columnwidth}\raggedleft
33\strut
\end{minipage} & \begin{minipage}[t]{0.11\columnwidth}\raggedleft
44\strut
\end{minipage}\tabularnewline
\begin{minipage}[t]{0.06\columnwidth}\raggedright
AEM\strut
\end{minipage} & \begin{minipage}[t]{0.45\columnwidth}\raggedright
Physiology\strut
\end{minipage} & \begin{minipage}[t]{0.03\columnwidth}\raggedleft
356\strut
\end{minipage} & \begin{minipage}[t]{0.08\columnwidth}\raggedleft
50.3\strut
\end{minipage} & \begin{minipage}[t]{0.11\columnwidth}\raggedleft
32\strut
\end{minipage} & \begin{minipage}[t]{0.11\columnwidth}\raggedleft
31\strut
\end{minipage}\tabularnewline
\begin{minipage}[t]{0.06\columnwidth}\raggedright
AEM\strut
\end{minipage} & \begin{minipage}[t]{0.45\columnwidth}\raggedright
Plant Microbiology\strut
\end{minipage} & \begin{minipage}[t]{0.03\columnwidth}\raggedleft
346\strut
\end{minipage} & \begin{minipage}[t]{0.08\columnwidth}\raggedleft
36.4\strut
\end{minipage} & \begin{minipage}[t]{0.11\columnwidth}\raggedleft
29\strut
\end{minipage} & \begin{minipage}[t]{0.11\columnwidth}\raggedleft
39\strut
\end{minipage}\tabularnewline
\begin{minipage}[t]{0.06\columnwidth}\raggedright
AEM\strut
\end{minipage} & \begin{minipage}[t]{0.45\columnwidth}\raggedright
Public and Environmental Health Microbiology\strut
\end{minipage} & \begin{minipage}[t]{0.03\columnwidth}\raggedleft
893\strut
\end{minipage} & \begin{minipage}[t]{0.08\columnwidth}\raggedleft
34.0\strut
\end{minipage} & \begin{minipage}[t]{0.11\columnwidth}\raggedleft
32\strut
\end{minipage} & \begin{minipage}[t]{0.11\columnwidth}\raggedleft
45\strut
\end{minipage}\tabularnewline
\begin{minipage}[t]{0.06\columnwidth}\raggedright
IAI\strut
\end{minipage} & \begin{minipage}[t]{0.45\columnwidth}\raggedright
Bacterial Infections\strut
\end{minipage} & \begin{minipage}[t]{0.03\columnwidth}\raggedleft
716\strut
\end{minipage} & \begin{minipage}[t]{0.08\columnwidth}\raggedleft
58.4\strut
\end{minipage} & \begin{minipage}[t]{0.11\columnwidth}\raggedleft
35\strut
\end{minipage} & \begin{minipage}[t]{0.11\columnwidth}\raggedleft
36\strut
\end{minipage}\tabularnewline
\begin{minipage}[t]{0.06\columnwidth}\raggedright
IAI\strut
\end{minipage} & \begin{minipage}[t]{0.45\columnwidth}\raggedright
Cellular Microbiology: Pathogen-Host Cell Molecular Interactions\strut
\end{minipage} & \begin{minipage}[t]{0.03\columnwidth}\raggedleft
685\strut
\end{minipage} & \begin{minipage}[t]{0.08\columnwidth}\raggedleft
55.2\strut
\end{minipage} & \begin{minipage}[t]{0.11\columnwidth}\raggedleft
35\strut
\end{minipage} & \begin{minipage}[t]{0.11\columnwidth}\raggedleft
37\strut
\end{minipage}\tabularnewline
\begin{minipage}[t]{0.06\columnwidth}\raggedright
IAI\strut
\end{minipage} & \begin{minipage}[t]{0.45\columnwidth}\raggedright
Fungal and Parasitic Infections\strut
\end{minipage} & \begin{minipage}[t]{0.03\columnwidth}\raggedleft
353\strut
\end{minipage} & \begin{minipage}[t]{0.08\columnwidth}\raggedleft
59.5\strut
\end{minipage} & \begin{minipage}[t]{0.11\columnwidth}\raggedleft
33\strut
\end{minipage} & \begin{minipage}[t]{0.11\columnwidth}\raggedleft
33\strut
\end{minipage}\tabularnewline
\begin{minipage}[t]{0.06\columnwidth}\raggedright
IAI\strut
\end{minipage} & \begin{minipage}[t]{0.45\columnwidth}\raggedright
Host Response and Inflammation\strut
\end{minipage} & \begin{minipage}[t]{0.03\columnwidth}\raggedleft
763\strut
\end{minipage} & \begin{minipage}[t]{0.08\columnwidth}\raggedleft
50.2\strut
\end{minipage} & \begin{minipage}[t]{0.11\columnwidth}\raggedleft
35\strut
\end{minipage} & \begin{minipage}[t]{0.11\columnwidth}\raggedleft
40\strut
\end{minipage}\tabularnewline
\begin{minipage}[t]{0.06\columnwidth}\raggedright
IAI\strut
\end{minipage} & \begin{minipage}[t]{0.45\columnwidth}\raggedright
Host-Associated Microbial Communities\strut
\end{minipage} & \begin{minipage}[t]{0.03\columnwidth}\raggedleft
7\strut
\end{minipage} & \begin{minipage}[t]{0.08\columnwidth}\raggedleft
57.1\strut
\end{minipage} & \begin{minipage}[t]{0.11\columnwidth}\raggedleft
43\strut
\end{minipage} & \begin{minipage}[t]{0.11\columnwidth}\raggedleft
86\strut
\end{minipage}\tabularnewline
\begin{minipage}[t]{0.06\columnwidth}\raggedright
IAI\strut
\end{minipage} & \begin{minipage}[t]{0.45\columnwidth}\raggedright
Microbial Immunity and Vaccines\strut
\end{minipage} & \begin{minipage}[t]{0.03\columnwidth}\raggedleft
342\strut
\end{minipage} & \begin{minipage}[t]{0.08\columnwidth}\raggedleft
56.4\strut
\end{minipage} & \begin{minipage}[t]{0.11\columnwidth}\raggedleft
35\strut
\end{minipage} & \begin{minipage}[t]{0.11\columnwidth}\raggedleft
32\strut
\end{minipage}\tabularnewline
\begin{minipage}[t]{0.06\columnwidth}\raggedright
IAI\strut
\end{minipage} & \begin{minipage}[t]{0.45\columnwidth}\raggedright
Molecular Genomics\strut
\end{minipage} & \begin{minipage}[t]{0.03\columnwidth}\raggedleft
33\strut
\end{minipage} & \begin{minipage}[t]{0.08\columnwidth}\raggedleft
60.6\strut
\end{minipage} & \begin{minipage}[t]{0.11\columnwidth}\raggedleft
37\strut
\end{minipage} & \begin{minipage}[t]{0.11\columnwidth}\raggedleft
33\strut
\end{minipage}\tabularnewline
\begin{minipage}[t]{0.06\columnwidth}\raggedright
IAI\strut
\end{minipage} & \begin{minipage}[t]{0.45\columnwidth}\raggedright
Molecular Pathogenesis\strut
\end{minipage} & \begin{minipage}[t]{0.03\columnwidth}\raggedleft
617\strut
\end{minipage} & \begin{minipage}[t]{0.08\columnwidth}\raggedleft
68.4\strut
\end{minipage} & \begin{minipage}[t]{0.11\columnwidth}\raggedleft
35\strut
\end{minipage} & \begin{minipage}[t]{0.11\columnwidth}\raggedleft
31\strut
\end{minipage}\tabularnewline
\begin{minipage}[t]{0.06\columnwidth}\raggedright
JCM\strut
\end{minipage} & \begin{minipage}[t]{0.45\columnwidth}\raggedright
Bacteriology\strut
\end{minipage} & \begin{minipage}[t]{0.03\columnwidth}\raggedleft
2952\strut
\end{minipage} & \begin{minipage}[t]{0.08\columnwidth}\raggedleft
33.2\strut
\end{minipage} & \begin{minipage}[t]{0.11\columnwidth}\raggedleft
27\strut
\end{minipage} & \begin{minipage}[t]{0.11\columnwidth}\raggedleft
41\strut
\end{minipage}\tabularnewline
\begin{minipage}[t]{0.06\columnwidth}\raggedright
JCM\strut
\end{minipage} & \begin{minipage}[t]{0.45\columnwidth}\raggedright
Chlamydiology and Rickettsiology\strut
\end{minipage} & \begin{minipage}[t]{0.03\columnwidth}\raggedleft
80\strut
\end{minipage} & \begin{minipage}[t]{0.08\columnwidth}\raggedleft
32.5\strut
\end{minipage} & \begin{minipage}[t]{0.11\columnwidth}\raggedleft
25\strut
\end{minipage} & \begin{minipage}[t]{0.11\columnwidth}\raggedleft
41\strut
\end{minipage}\tabularnewline
\begin{minipage}[t]{0.06\columnwidth}\raggedright
JCM\strut
\end{minipage} & \begin{minipage}[t]{0.45\columnwidth}\raggedright
Clinical Veterinary Microbiology\strut
\end{minipage} & \begin{minipage}[t]{0.03\columnwidth}\raggedleft
364\strut
\end{minipage} & \begin{minipage}[t]{0.08\columnwidth}\raggedleft
32.7\strut
\end{minipage} & \begin{minipage}[t]{0.11\columnwidth}\raggedleft
29\strut
\end{minipage} & \begin{minipage}[t]{0.11\columnwidth}\raggedleft
40\strut
\end{minipage}\tabularnewline
\begin{minipage}[t]{0.06\columnwidth}\raggedright
JCM\strut
\end{minipage} & \begin{minipage}[t]{0.45\columnwidth}\raggedright
Epidemiology\strut
\end{minipage} & \begin{minipage}[t]{0.03\columnwidth}\raggedleft
854\strut
\end{minipage} & \begin{minipage}[t]{0.08\columnwidth}\raggedleft
29.7\strut
\end{minipage} & \begin{minipage}[t]{0.11\columnwidth}\raggedleft
30\strut
\end{minipage} & \begin{minipage}[t]{0.11\columnwidth}\raggedleft
45\strut
\end{minipage}\tabularnewline
\begin{minipage}[t]{0.06\columnwidth}\raggedright
JCM\strut
\end{minipage} & \begin{minipage}[t]{0.45\columnwidth}\raggedright
Fast-Track Communications\strut
\end{minipage} & \begin{minipage}[t]{0.03\columnwidth}\raggedleft
5\strut
\end{minipage} & \begin{minipage}[t]{0.08\columnwidth}\raggedleft
40.0\strut
\end{minipage} & \begin{minipage}[t]{0.11\columnwidth}\raggedleft
33\strut
\end{minipage} & \begin{minipage}[t]{0.11\columnwidth}\raggedleft
40\strut
\end{minipage}\tabularnewline
\begin{minipage}[t]{0.06\columnwidth}\raggedright
JCM\strut
\end{minipage} & \begin{minipage}[t]{0.45\columnwidth}\raggedright
Immunoassays\strut
\end{minipage} & \begin{minipage}[t]{0.03\columnwidth}\raggedleft
139\strut
\end{minipage} & \begin{minipage}[t]{0.08\columnwidth}\raggedleft
36.0\strut
\end{minipage} & \begin{minipage}[t]{0.11\columnwidth}\raggedleft
31\strut
\end{minipage} & \begin{minipage}[t]{0.11\columnwidth}\raggedleft
41\strut
\end{minipage}\tabularnewline
\begin{minipage}[t]{0.06\columnwidth}\raggedright
JCM\strut
\end{minipage} & \begin{minipage}[t]{0.45\columnwidth}\raggedright
Mycobacteriology and Aerobic Actinomycetes\strut
\end{minipage} & \begin{minipage}[t]{0.03\columnwidth}\raggedleft
510\strut
\end{minipage} & \begin{minipage}[t]{0.08\columnwidth}\raggedleft
42.9\strut
\end{minipage} & \begin{minipage}[t]{0.11\columnwidth}\raggedleft
32\strut
\end{minipage} & \begin{minipage}[t]{0.11\columnwidth}\raggedleft
41\strut
\end{minipage}\tabularnewline
\begin{minipage}[t]{0.06\columnwidth}\raggedright
JCM\strut
\end{minipage} & \begin{minipage}[t]{0.45\columnwidth}\raggedright
Mycology\strut
\end{minipage} & \begin{minipage}[t]{0.03\columnwidth}\raggedleft
587\strut
\end{minipage} & \begin{minipage}[t]{0.08\columnwidth}\raggedleft
37.3\strut
\end{minipage} & \begin{minipage}[t]{0.11\columnwidth}\raggedleft
19\strut
\end{minipage} & \begin{minipage}[t]{0.11\columnwidth}\raggedleft
39\strut
\end{minipage}\tabularnewline
\begin{minipage}[t]{0.06\columnwidth}\raggedright
JCM\strut
\end{minipage} & \begin{minipage}[t]{0.45\columnwidth}\raggedright
Parasitology\strut
\end{minipage} & \begin{minipage}[t]{0.03\columnwidth}\raggedleft
337\strut
\end{minipage} & \begin{minipage}[t]{0.08\columnwidth}\raggedleft
33.2\strut
\end{minipage} & \begin{minipage}[t]{0.11\columnwidth}\raggedleft
27\strut
\end{minipage} & \begin{minipage}[t]{0.11\columnwidth}\raggedleft
34\strut
\end{minipage}\tabularnewline
\begin{minipage}[t]{0.06\columnwidth}\raggedright
JCM\strut
\end{minipage} & \begin{minipage}[t]{0.45\columnwidth}\raggedright
Virology\strut
\end{minipage} & \begin{minipage}[t]{0.03\columnwidth}\raggedleft
1140\strut
\end{minipage} & \begin{minipage}[t]{0.08\columnwidth}\raggedleft
37.5\strut
\end{minipage} & \begin{minipage}[t]{0.11\columnwidth}\raggedleft
29\strut
\end{minipage} & \begin{minipage}[t]{0.11\columnwidth}\raggedleft
41\strut
\end{minipage}\tabularnewline
\begin{minipage}[t]{0.06\columnwidth}\raggedright
JVI\strut
\end{minipage} & \begin{minipage}[t]{0.45\columnwidth}\raggedright
Cellular Response to Infection\strut
\end{minipage} & \begin{minipage}[t]{0.03\columnwidth}\raggedleft
604\strut
\end{minipage} & \begin{minipage}[t]{0.08\columnwidth}\raggedleft
51.2\strut
\end{minipage} & \begin{minipage}[t]{0.11\columnwidth}\raggedleft
36\strut
\end{minipage} & \begin{minipage}[t]{0.11\columnwidth}\raggedleft
32\strut
\end{minipage}\tabularnewline
\begin{minipage}[t]{0.06\columnwidth}\raggedright
JVI\strut
\end{minipage} & \begin{minipage}[t]{0.45\columnwidth}\raggedright
Gene Delivery\strut
\end{minipage} & \begin{minipage}[t]{0.03\columnwidth}\raggedleft
98\strut
\end{minipage} & \begin{minipage}[t]{0.08\columnwidth}\raggedleft
41.8\strut
\end{minipage} & \begin{minipage}[t]{0.11\columnwidth}\raggedleft
32\strut
\end{minipage} & \begin{minipage}[t]{0.11\columnwidth}\raggedleft
20\strut
\end{minipage}\tabularnewline
\begin{minipage}[t]{0.06\columnwidth}\raggedright
JVI\strut
\end{minipage} & \begin{minipage}[t]{0.45\columnwidth}\raggedright
Genetic Diversity and Evolution\strut
\end{minipage} & \begin{minipage}[t]{0.03\columnwidth}\raggedleft
883\strut
\end{minipage} & \begin{minipage}[t]{0.08\columnwidth}\raggedleft
51.1\strut
\end{minipage} & \begin{minipage}[t]{0.11\columnwidth}\raggedleft
39\strut
\end{minipage} & \begin{minipage}[t]{0.11\columnwidth}\raggedleft
27\strut
\end{minipage}\tabularnewline
\begin{minipage}[t]{0.06\columnwidth}\raggedright
JVI\strut
\end{minipage} & \begin{minipage}[t]{0.45\columnwidth}\raggedright
Genome Replication and Regulation of Viral Gene Expression\strut
\end{minipage} & \begin{minipage}[t]{0.03\columnwidth}\raggedleft
813\strut
\end{minipage} & \begin{minipage}[t]{0.08\columnwidth}\raggedleft
64.6\strut
\end{minipage} & \begin{minipage}[t]{0.11\columnwidth}\raggedleft
39\strut
\end{minipage} & \begin{minipage}[t]{0.11\columnwidth}\raggedleft
23\strut
\end{minipage}\tabularnewline
\begin{minipage}[t]{0.06\columnwidth}\raggedright
JVI\strut
\end{minipage} & \begin{minipage}[t]{0.45\columnwidth}\raggedright
Pathogenesis and Immunity\strut
\end{minipage} & \begin{minipage}[t]{0.03\columnwidth}\raggedleft
1622\strut
\end{minipage} & \begin{minipage}[t]{0.08\columnwidth}\raggedleft
60.4\strut
\end{minipage} & \begin{minipage}[t]{0.11\columnwidth}\raggedleft
35\strut
\end{minipage} & \begin{minipage}[t]{0.11\columnwidth}\raggedleft
33\strut
\end{minipage}\tabularnewline
\begin{minipage}[t]{0.06\columnwidth}\raggedright
JVI\strut
\end{minipage} & \begin{minipage}[t]{0.45\columnwidth}\raggedright
Prions\strut
\end{minipage} & \begin{minipage}[t]{0.03\columnwidth}\raggedleft
92\strut
\end{minipage} & \begin{minipage}[t]{0.08\columnwidth}\raggedleft
69.6\strut
\end{minipage} & \begin{minipage}[t]{0.11\columnwidth}\raggedleft
56\strut
\end{minipage} & \begin{minipage}[t]{0.11\columnwidth}\raggedleft
22\strut
\end{minipage}\tabularnewline
\begin{minipage}[t]{0.06\columnwidth}\raggedright
JVI\strut
\end{minipage} & \begin{minipage}[t]{0.45\columnwidth}\raggedright
Structure and Assembly\strut
\end{minipage} & \begin{minipage}[t]{0.03\columnwidth}\raggedleft
725\strut
\end{minipage} & \begin{minipage}[t]{0.08\columnwidth}\raggedleft
71.3\strut
\end{minipage} & \begin{minipage}[t]{0.11\columnwidth}\raggedleft
39\strut
\end{minipage} & \begin{minipage}[t]{0.11\columnwidth}\raggedleft
29\strut
\end{minipage}\tabularnewline
\begin{minipage}[t]{0.06\columnwidth}\raggedright
JVI\strut
\end{minipage} & \begin{minipage}[t]{0.45\columnwidth}\raggedright
Transformation and Oncogenesis\strut
\end{minipage} & \begin{minipage}[t]{0.03\columnwidth}\raggedleft
154\strut
\end{minipage} & \begin{minipage}[t]{0.08\columnwidth}\raggedleft
59.1\strut
\end{minipage} & \begin{minipage}[t]{0.11\columnwidth}\raggedleft
39\strut
\end{minipage} & \begin{minipage}[t]{0.11\columnwidth}\raggedleft
36\strut
\end{minipage}\tabularnewline
\begin{minipage}[t]{0.06\columnwidth}\raggedright
JVI\strut
\end{minipage} & \begin{minipage}[t]{0.45\columnwidth}\raggedright
Vaccines and Antiviral Agents\strut
\end{minipage} & \begin{minipage}[t]{0.03\columnwidth}\raggedleft
1149\strut
\end{minipage} & \begin{minipage}[t]{0.08\columnwidth}\raggedleft
59.2\strut
\end{minipage} & \begin{minipage}[t]{0.11\columnwidth}\raggedleft
36\strut
\end{minipage} & \begin{minipage}[t]{0.11\columnwidth}\raggedleft
28\strut
\end{minipage}\tabularnewline
\begin{minipage}[t]{0.06\columnwidth}\raggedright
JVI\strut
\end{minipage} & \begin{minipage}[t]{0.45\columnwidth}\raggedright
Virus-Cell Interactions\strut
\end{minipage} & \begin{minipage}[t]{0.03\columnwidth}\raggedleft
2414\strut
\end{minipage} & \begin{minipage}[t]{0.08\columnwidth}\raggedleft
63.6\strut
\end{minipage} & \begin{minipage}[t]{0.11\columnwidth}\raggedleft
40\strut
\end{minipage} & \begin{minipage}[t]{0.11\columnwidth}\raggedleft
30\strut
\end{minipage}\tabularnewline
\bottomrule
\end{longtable}

\newpage

\textbf{Figure 1. Overview of manuscript outcomes.} 108,706 manuscript
records were obtained for the period between January 2012 and August
2018. After eliminating non-primary research manuscripts and linking
records for resubmitted manuscripts, we processed 79,189 unique
manuscripts. The median number of versions was 1 (IQR=0-2) with a median
of 6 (IQR=1-11) authors per manuscript. As of August 2018, 34,196 of
these were published at the ASM journals. Revisions were requested for
24,016 manuscripts and 53,436 manuscripts were rejected at their first
submission. The number of individuals (e.g., author, editor, reviewer)
involved in each category of manuscript decision are indicated in the
colored boxes: women (orange), men (blue), and unknown (gray). *A small
number were given revise (242) or acceptance (1094) decisions without
review.

\textbf{Figure 2. Gendered representation among gatekeepers.} Proportion
of editors from (A) institution types and (B) over time. Editors and
senior editors are pooled together. (C) The proportion of editors (solid
line) and their workloads (dashed line) at each of the ASM journals from
2012 to 2018. Proportion of reviewers from (D) institution types and (E)
over time. (A,D) Each gender equals 100\% when all institutions are
summed. The total number of gatekeepers from the indicated institution
are in parentheses. (B,E) Each individual was counted once per calendar
year, proportions of each gender add to 100\% per year.

\textbf{Figure 3. Gatekeeper workload and response to requests to
review.} (A) Proportion of manuscript workload handled by men and women
editors, editorial rejections excluded. (B) Box plot comparison of all
manuscripts by reviewer gender on a log\textsubscript{10} scale. (C) The
percent of reviewers by gender that accepted the opportunity to review,
split according to the editor's gender.

\textbf{Figure 4. Author representation by gender.} The proportion of
(A) men, women, and unknown gender senior authors from each institution
type (where the number of authors are in parentheses), (B) men, women,
and unknown (senior and co-) authors from 2012 - 2018. Each individual
was counted once per calendar year. The proportion of (C) first authors
and (D) corresponding authors from 2012 - 2018 on submitted manuscripts
(dashed line) and published papers (solid line).

\textbf{Figure 5. Difference in manuscript outcomes by author gender.}
The difference in the percent of manuscript outcomes was calculated by
subtracting the percent of women who received the outcome, from the
percent of men. A negative value (orange) indicates women received the
outcome more often, 0 (or no bar) indicates equal rates of the outcome,
and positive (blue) indicates that men received the outcome more
frequently. Vertical lines indicate the difference value for all
journals combined. (A) The difference in percent rejections by author
gender and type (e.g., corresponding, first, last, middle) at any stage
across all journals. (B) The difference in percent editorial rejection
rates for corresponding authors at each journal. (C) The difference in
percentage points between each decision type for corresponding authors
following the first peer review.

\textbf{Figure 6. Difference in decisions or recommendations according
to the gatekeeper gender.} The difference in the percent of manuscript
outcomes was calculated by subtracting the percent of women who received
the outcome, from the percent of men. A negative value (orange)
indicates women received the outcome more often, 0 (or no bar) indicates
equal rates of the outcome, and positive (blue) indicates that men
received the outcome more frequently. (A) Effect of editor gender on the
difference in decisions following review. (B) Difference in percentage
points for review recommendations and (C) how that is affected by
reviewer gender. (A-C) All journals combined.

\textbf{Figure 7. Impact of origin and US institution type on manuscript
decisions by gender.} The difference in the percent of manuscript
outcomes was calculated by subtracting the percent of women who received
the outcome, from the percent of men. A negative value (orange)
indicates women received the outcome more often, 0 (or no bar) indicates
equal rates of the outcome, and positive (blue) indicates that men
received the outcome more frequently. Vertical lines indicate the
difference value for all of the ASM journals combined. Difference in
percentage points for (A) editorial rejections and (B) following first
review of manuscripts submitted by US-based corresponding authors.
Vertical line indicates value for all of the ASM journals combined. (C)
Difference in percentage points for acceptance and editorial rejections
according to institution types and (D) acceptance decisions by editor
gender and institution type.

\textbf{Figure S1.} (A) Equation for calculating negative bias by
genderize algorithm. C indicates a country. (B) The negative impact of
each country on the overall gender inference of the full data-set.
Number to the right of each column is the total number of names
associated with that country.

\textbf{Figure S2.} The proportion of (A) potential reviewers at all of
the ASM journals combined, (B) reviewers at each ASM journal.

\textbf{Figure S3.} The proportion of all submitted (dashed line) and
published (solid line) (A) middle and (B) last authors by gender at each
ASM journal.

\textbf{Figure S4.} The proportion of women authors (x-axis) on
submitted manuscripts (y axis, log\textsubscript{10} scale) according to
the number of co-authors (individual plot) and the gender of the
corresponding author (orange/blue). Single author papers were eliminated
and the manuscripts grouped according to the number of co-authors:
groups of 5 authors up to 20, groups of 10 authors up to 40, and a
single group of manuscripts with 40+ authors. The manuscripts were then
split according to the proportion of co-authors that were inferred to be
women: 0, up to 24\%, 25-50\%, 51-75\%, and more than 75\%. The
manuscripts were then further split according to the inferred gender of
the corresponding author. Regardless of the number of co-authors, women
corresponding authors (orange) submitted more manuscripts with a
majority (\textgreater50\%) of women co-authors than men corresponding
authors. Men corresponding authors submitted more manuscripts with less
than 50\% women co-authors than women corresponding authors did, and the
trend of this gap increased with the number of co-authors.

\textbf{Figure S5.} Boxplots of linear regression results from 25
data-splits. A) AUC values, each panel represents a different prediction
model: A - Corresponding author's gender; B - Author gender on editorial
decisions; C - Institution on editorial decisions. B,C) Effect of
variables on the logistic regression model represented as the absolute
values of the variable weight. B) Author gender from editorial
decisions. C) Corresponding author's gender.

\textbf{Figure S6.} Comparison of time to final accepted decision and
time spent in the peer review system by journal and gender. The number
of days (A) between when a manuscript is initially submitted and
officially accepted or (B) that a manuscript spends in the ASM peer
review system (i.e., the sum of days from author submission to editor
decision for each submitted version).

\textbf{Figure S7.} Difference in A) editorial rejection and B)
acceptance rates by journal and institution type. C) Difference in
review recommendations by reviewer gender and author institution type.
D) Median importance (black dot) of features affecting editorial
rejections, and their range. Color of smaller dots (N=25) indicate the
direction of the impact.

\textbf{Figure S8.} Difference in editorial rejections and rejections
after review by corresponding author gender and manuscript category at
(A) AAC, (B) AEM, (C) IAI, (D) JCM, and (E) JVI. In parentheses: N = the
number of manuscripts submitted.


\end{document}
