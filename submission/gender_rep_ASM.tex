\documentclass[11pt,]{article}
\usepackage{lmodern}
\usepackage{amssymb,amsmath}
\usepackage{ifxetex,ifluatex}
\usepackage{fixltx2e} % provides \textsubscript
\ifnum 0\ifxetex 1\fi\ifluatex 1\fi=0 % if pdftex
  \usepackage[T1]{fontenc}
  \usepackage[utf8]{inputenc}
\else % if luatex or xelatex
  \ifxetex
    \usepackage{mathspec}
  \else
    \usepackage{fontspec}
  \fi
  \defaultfontfeatures{Ligatures=TeX,Scale=MatchLowercase}
\fi
% use upquote if available, for straight quotes in verbatim environments
\IfFileExists{upquote.sty}{\usepackage{upquote}}{}
% use microtype if available
\IfFileExists{microtype.sty}{%
\usepackage{microtype}
\UseMicrotypeSet[protrusion]{basicmath} % disable protrusion for tt fonts
}{}
\usepackage[margin=1.0in]{geometry}
\usepackage{hyperref}
\hypersetup{unicode=true,
            pdftitle={Who are ASM Journals? A Gender Analysis},
            pdfborder={0 0 0},
            breaklinks=true}
\urlstyle{same}  % don't use monospace font for urls
\usepackage{graphicx,grffile}
\makeatletter
\def\maxwidth{\ifdim\Gin@nat@width>\linewidth\linewidth\else\Gin@nat@width\fi}
\def\maxheight{\ifdim\Gin@nat@height>\textheight\textheight\else\Gin@nat@height\fi}
\makeatother
% Scale images if necessary, so that they will not overflow the page
% margins by default, and it is still possible to overwrite the defaults
% using explicit options in \includegraphics[width, height, ...]{}
\setkeys{Gin}{width=\maxwidth,height=\maxheight,keepaspectratio}
\IfFileExists{parskip.sty}{%
\usepackage{parskip}
}{% else
\setlength{\parindent}{0pt}
\setlength{\parskip}{6pt plus 2pt minus 1pt}
}
\setlength{\emergencystretch}{3em}  % prevent overfull lines
\providecommand{\tightlist}{%
  \setlength{\itemsep}{0pt}\setlength{\parskip}{0pt}}
\setcounter{secnumdepth}{0}
% Redefines (sub)paragraphs to behave more like sections
\ifx\paragraph\undefined\else
\let\oldparagraph\paragraph
\renewcommand{\paragraph}[1]{\oldparagraph{#1}\mbox{}}
\fi
\ifx\subparagraph\undefined\else
\let\oldsubparagraph\subparagraph
\renewcommand{\subparagraph}[1]{\oldsubparagraph{#1}\mbox{}}
\fi

%%% Use protect on footnotes to avoid problems with footnotes in titles
\let\rmarkdownfootnote\footnote%
\def\footnote{\protect\rmarkdownfootnote}

%%% Change title format to be more compact
\usepackage{titling}

% Create subtitle command for use in maketitle
\newcommand{\subtitle}[1]{
  \posttitle{
    \begin{center}\large#1\end{center}
    }
}

\setlength{\droptitle}{-2em}

  \title{\textbf{Who are ASM Journals? A Gender Analysis}}
    \pretitle{\vspace{\droptitle}\centering\huge}
  \posttitle{\par}
    \author{}
    \preauthor{}\postauthor{}
    \date{}
    \predate{}\postdate{}
  
\usepackage{booktabs}
\usepackage{longtable}
\usepackage{array}
\usepackage{multirow}
\usepackage[table]{xcolor}
\usepackage{wrapfig}
\usepackage{float}
\usepackage{colortbl}
\usepackage{pdflscape}
\usepackage{tabu}
\usepackage{threeparttable}
\usepackage{threeparttablex}
\usepackage[normalem]{ulem}
\usepackage{makecell}
\usepackage{caption}

\usepackage{helvet} % Helvetica font
\renewcommand*\familydefault{\sfdefault} % Use the sans serif version of the font
\usepackage[T1]{fontenc}

\usepackage[none]{hyphenat}

\usepackage{setspace}
\doublespacing
\setlength{\parskip}{1em}

\usepackage{lineno}

\usepackage{pdfpages}
\floatplacement{figure}{H} % Keep the figure up top of the page
\usepackage{booktabs}
\usepackage{longtable}
\usepackage{array}
\usepackage{multirow}
\usepackage{wrapfig}
\usepackage{float}
\usepackage{colortbl}
\usepackage{pdflscape}
\usepackage{tabu}
\usepackage{threeparttable}
\usepackage{threeparttablex}
\usepackage[normalem]{ulem}
\usepackage{makecell}
\usepackage{xcolor}

\begin{document}
\maketitle

\vspace{35mm}

Running title: A gender analysis of ASM journals

\vspace{35mm}

Ada K. Hagan\({^1}\), Begüm D. Topçuoğlu\({^1}\), Hazel Barton\({^2}\),
Patrick D. Schloss\textsuperscript{1\(\dagger\)}

\vspace{40mm}

\(\dagger\) To whom correspondence should be addressed:
\href{mailto:pschloss@umich.edu}{\nolinkurl{pschloss@umich.edu}}

1. Department of Microbiology and Immunology, University of Michigan,
Ann Arbor, MI 48109

2. Department of Biology, University of Akron, Akron, OH

\newpage

\linenumbers

\subsection{Abstract}\label{abstract}

\subsection{Importance}\label{importance}

\subsection{Introduction}\label{introduction}

Scientific societies play an integral role in the formation and
maintanence of scientific communities. They host conferences that
provide a forum for knowledge exchange and networking, as well as
opportunities for increased visibility as a researcher. Scientific
societies also frequently publish the most reputable journals in their
field, facilitating the peer review process to vet new research
submissions. As such, societies have great power to set both
professional and scientfic norms in their community by choosing what
behaviors are rewarded and what types of research are accepted for
publication. Authorship is a coveted measure of success in academic
research as it is a key criterium for hiring and promotion processes.
Accordingly, editors and reviewers of research journals have a
substantial influence over the futures of hopeful authors. While the
membership of scientific societies is likely to reflective of all those
who participate in the field, regardless of career track, the
gatekeepers for peer review (reviewers and editors) are more reflective
of the academy than the society as a whole.

Evidence has accumulated over the decades that academic research has a
representation problem. While at least 50\% of biology Ph.D.~graduates
are women, the number of women in postdoctoral positions and
tenure-track positions are less than 40 and 30\%, respectively
@article\{sheltzer\_elite\_2014\}. Studies examining other metrics such
as race and ethnicity find that less than 10\% of all science and
engineering doctorates were awarded to underrepresented minorities,
while less than 25\% of science and engineering doctorates in early
career academia identify as non-white (NSF ADVANCE, 2014). Predictabily,
the disparities increase alongside academic rank, a phenomenon known as
hierarchical segregation @article\{potvin\_diversity\_2018\}. There have
been many proposed reasons for these disparities (particularly against
women) that include biases in training and hiring, the impact of
children on career trajectories, a lack of support for primary
caregivers, and a lack of recognition, which culminate in reduced
productivity as measured by research publications\textbf{Add citations}.

Recently, scientific societies and publishers have begun examining their
own data to evaluate representation of, and bias against, women and
minorities in their peer review processes. The American Geological Union
found that while the acceptance rate of women-authored publications was
greater than that for publications authored by men, women submitted
fewer manuscripts than men and were used as reviewers only 20\% of the
time (Lerback, 2017). Fox et al., have found that for the journal
\emph{Functional Ecology}, the proportion of women invited to review
depended on the gender of the editor (Fox, 2016). Despite the
disproportional representation of lead women authors, several studies
have concluded that there is no significant bias aginst papers authored
by women (C\&W, 2011; Fox, 2016; Handly, 2015; Edwards, 2018).
Conversely, two recent studies---one of the peer review process at eLife
and the other of outcomes at six ecology and evolution journals---found
that women-authored papers are less likely to have positive reviews and
outcomes (Murray, 2018; Fox and Paine, 2019).

However, representation and attitudes differ by scientific field and no
studies to-date seem to have investigated academic publishing in the
field of microbiology. The American Society for Microbiology (ASM) is
one of the largest life science societies, with an average membership of
41,000 since 1990. In its mission statement, the ASM notes that it is
``an inclusive organization, engaging with and responding to the needs
of its diverse constituencies'' and pledges to ``address all members'
needs through development and assessment of programs and services.'' One
of these services is the publication of microbiology research through a
suite of 13 journals. Led by the ASM Journals Department, these journals
boast of ``quality peer review and editorial leadership.'' As bastions
of the microbiology field, these journals are historically responsible
for the success of microbiologists. The goal of this research study is
two-fold: first, to understand the representation of authors, reviewers,
and editors; second, to examine the possibility of gender bias in peer
review at ASM journals.

\subsection{Results}\label{results}

This study uses the data from manuscripts submitted to ASM journals that
had a final decision rendered between January 2012 and August 2018.

\emph{Men dominate as gatekeepers and senior authors} + Editor Notes: +
The proportions of men/women editors are similar for both editors and
senior editors -- only showed editors + There is a slow trend toward
gender parity in editors + Need to add the N of each editorial group:
senior editors, editors, eic -- all are pooled

\begin{verbatim}
+ Who carries the editorial burden?
   + How many manuscripts are editors handling -- Women handle a higher median of papers than men, across ASM. Varies by journal
   + Is it proportional to the author representation? -- in some cases, but also difficult to compare b/c of the high "unclear" proportion of authors
   + Does it change by EiC? -- yes, CVI/JVI have EiCs that are women
\end{verbatim}

\includegraphics{Figure_1.png}

\textbf{Figure 1. Gendered representation among editors.} Proportion of
editors (solid line) and their workload (dashed lines) from 2012 to
2018. Data for men are blue, and women orange. (A) All journals
combined. (B) Breakdown by individual journals. Editors and senior
editors are pooled together, editorial rejections are excluded. Each
individual was counted once per calendar year.

\begin{itemize}
\tightlist
\item
  Reviewer Notes:

  \begin{itemize}
  \tightlist
  \item
    Median number of paper reviews are equivalent across genders, though
    men seem slightly more likely to review more papers - trends
    representative of journals, except for MCB where ``unclear''
    reviewers are a greater proportion than women
  \item
    Slight increase in the proportion of women over time (\& decrease of
    men), more women reviewers than ``unclear''
  \item
    trends and approximate proportions are equvalent for potential
    reviewers and actual reviewers
  \item
    greater \# of women (unique indvidiuals) have been suggested as
    potential reviewers than \# of women senior authors -- only 1/2 have
    actually been reviewers
  \item
    similar trend for min, except that the proportions of potential \&
    accepted seem to be higher - calculate \& add proportions relative
    to senior authors
  \item
    different trend for ``unclear'' authors where senior \textgreater{}
    potential \textgreater{} accepted
  \end{itemize}
\item
  Who carries the burden of peer review?

  \begin{itemize}
  \tightlist
  \item
    How often are reviewers participating on multiple manuscripts? --
    calculate proportion that only reviewed one paper
  \item
    How does representation compare from potential to actual reviewer?
    -- seems equivalent
  \item
    Is it proportional to the author representation? -- yes
  \end{itemize}
\end{itemize}

\includegraphics{Figure_2.png}

\textbf{Figure 2. Reviewer representation, workload, and response to
requests to review.} (A) Proportion of each gender listed as a possible
reviewer from 2012 to 2018. (B) Comparison of total papers reviewed by
each individual according to gender. (C) Percent of each reviewer gender
contacted to review, according to the editor's gender. (D) The percent
of reviewers by gender that either accepted the opportunity to review or
did not respond to a request to review, split according to the editor's
gender. Reviewers were assigned one of three genders: men (blue,
dashed), women (orange, solid), or unclear (gray, dotted). Each
individual was counted once per calendar year.

\begin{itemize}
\tightlist
\item
  Notes:

  \begin{itemize}
  \tightlist
  \item
    Proportions of men and women authors have decreased over time, while
    ``unclear'' has increased -- gap between men \& women proportions
    unchanged since 2012
  \item
    ratio of women:men authors is 3:4
  \item
    First \& Middle author proportions are representative
  \item
    Last \& corresponding author proportions are representative
  \item
    Women are first authors more often than other groups -- decreasing
    trend of men/women first authors over time (increase of unclear)
  \item
    Trends are similar across journals \& genders
  \item
    proportion of women corresponding/last authors has not increased
    over time
  \item
    proportion of men senior authors has decreased over time -- increase
    of unclear senior authors
  \item
    proportion of published is higher for both men \& women than
    submitted proportions -- means that proportion of unclear published
    is lower than submitted
  \item
    is the increase in published proportions, proportionally equivalent
    for men \& women?
  \end{itemize}
\end{itemize}

\includegraphics{Figure_3.png}

\textbf{Figure 3. Author representation by gender.} The proportion of
(A) unique authors, (B) first authors, and (C) corresponding authors
from 2012 - 2018. (A, B) Solid lines indicate individuals, dashed
indicate proportion of manuscripts submitted. Men indicated by blue and
women by orange. All individuals counted once per calendar year. The
proportion of women authors on submitted papers according to (D) the
gender of the corresponding author or (E) the number of authors. Unique
manuscripts submitted from 2012 to 2018.

\emph{Women are less likely to be retained through stages of the peer
review system at ASM journals} + How does gender representation of
manuscripts compare from submitted to published? + proportion of M \& W
authored publications seems relatively disparate from M \& W authored
submissions -- unknown if bias + ``Publication Rate Disparity'' = (M
publish rate) - (W publish rate) -- positive == men more likely to
publish (i.e., overperform); negative == women more likely to publish +
Men overperform in all authorship roles across ASM journals, most clear
for corresponding authors + Break it down by journals and there is clear
trend to overperformance by men in both corresponding \& last
authorships, some exceptions + Are women being retained through peer
review at the same proportion as men/unclear? + Get actual proportions
at each stage + seems as if fewer women progress through each stage than
men + large proportion of potential reviewers are not senior authors
(both genders) + senior authors that are women are less likely to be
considered as reviewers (40 vs 50\%)

\includegraphics{Figure_4.png}

\textbf{Figure 4. The difference in percentage points of papers
accepted} at (A) all journals or (B) for each journal. Unique
manuscripts were split according to the gender of the corresponding,
first, last, and middle author(s), and the acceptance rate for each
group calculated. The difference in acceptance rate was determined by
subtracting the acceptance rate of women-authored papers from
men-authored papers. The shade (ranging from orange to blue) indicates
the outperforming gender. No bar indicates no difference in percentage
points.

\includegraphics{Figure_5.png}

\textbf{Figure 5. The retention of each gender through the publishing
roles.} All junior (first or middle) authors were split by gender and
tracked through their roles in academic publishing from senior author
(last or corresponding), potential reviewer (considered), reviewer
(accepted), or editor. Color indicates whether (purple) or not (green)
the individual participiated in that role at any point from 2012 to
2018.

\emph{Papers submitted by women are rejected more frequently than those
submitted by men}

\includegraphics{Figure_6.png}

\textbf{Figure 6. Difference in rejection rates by author gender.} The
percent of manuscripts rejected by author gender and type (e.g.,
corresponding, first, last, middle) at (A) all journals combined or at
(B) each journal, which shows the difference in percent rejection rates.
(C) The difference in percent editorial rejection rates at each journal,
vertical line indicates the difference for all journals combined. (D)
The difference in percentage points between each decision type following
the first peer review, vertical lines indicate the difference value for
all journals combined. The difference in rejection rates was determined
by subtracting the rejection rate of women-authored papers from
men-authored papers within each category. The shade (ranging from orange
to blue) indicates the outperforming gender. No bar indicates no
difference in percentange points.

\begin{itemize}
\tightlist
\item
  try to better understand the \% point difference in gender-ed
  performance
\item
  NOTE: to correct for the large discrepency in the participation of
  women at ASM journals during the is time period, all comparisions are
  made relative to the gender and population in question.
\item
  direct comparision of the proportion of woman-authored papers that are
  rejected compared to men, at each author stage

  \begin{itemize}
  \tightlist
  \item
    middle/first author rejected at similar rates
  \item
    woman-authored (corres/last) rejected more frequently than men
  \item
    there are several journals where the overall trend is repeated
    and/or amplified (e.g., AAC, IAI, JB, mBio, MCB)
  \item
    greatest effect seen for corresponding authors, use this
    sub-population to further examine
  \end{itemize}
\end{itemize}

\emph{Both men and women editors are more likely to reject papers
authored by women than those authored by men and these preferences are
partially mediated by author institution}

\includegraphics{Figure_7.png}

\textbf{Figure 7. Difference in decisions or reccomendations according
to the gatekeeper gender.} (A) Effect of editor gender on the difference
in percentage points for decisions following review at all journals
combined. (B) Difference in percentage points for review recommendations
and (C) how that is affected by reviewer gender.

\textbf{need to connect fig 7 institution info with the story -- stats
on gender?-- SUPPLEMENTAL}

\begin{itemize}
\tightlist
\item
  There are two main gatekeeping roles for manuscript decisions: editors
  \& reviewers
\item
  This section evaluates disparities made by editors during editorial
  rejections, and decisions following peer review.
\item
  Women recieve more editorial rejections than men, proportionally
  (editorial\_rejections.R)
\item
  may be complications due to geographic bias (eLife paper), restrict to
  papers submitted in US \& examine institution type
\item
  slight disparity in editorial rejection \& acceptance rates according
  to institution type (rej\_by\_inst\_type.R)

  \begin{itemize}
  \item
    greatest difference (4\%) occurs for institute/medical schools
  \item
    trend holds for most journals and is \textgreater{}20\% diff at JCM
    (R2s), mBio (Federal), MCB (low research), AEM (industry research)
    (Supplementary\_A)
  \item
    men from institutes/medical schools outperform women
    \textgreater{}7\% for acceptance across all journals
  \item
    \begin{quote}
    20\% in favor of men at: EC (fed researh), mSphere \& JVI (Low
    research), AEM \& MCB (industry research), JCM/JB/JCM (R2
    institution) (Supplementary\_B)
    \end{quote}
  \end{itemize}
\item
  both men \& women editors are more likely to reject women
  (editor\_gender\_analysis.R)

  \begin{itemize}
  \tightlist
  \item
    men editors are more likely to make revise only \& accept decisions
    for men
  \item
    women editors highly favor men from medical schools, slightly favor
    men from R2 \& industry -- what is the n?
  \item
    men editors favor men from R1, R2 and medical schools, slightly
    favor women from low \& industry research -- what is the n?
  \end{itemize}
\item
  rev\_score\_analysis.R

  \begin{itemize}
  \tightlist
  \item
    reviewers are more likely to suggest rejections for women as
    compared to men. No difference in revise decisions (A)
  \item
    men at R1, low research, medical schools \& fed research are favored
    by reviewers over women (B)
  \end{itemize}
\item
  reviewer\_gender\_analysis.R

  \begin{itemize}
  \tightlist
  \item
    both female \& male reviewers are more likely to recommend rejection
    for women. (C)
  \item
    Male reviewers are more likely to accept papers from men (C)
  \item
    women reviewers are more likely to recommend acceptance for women
    from low research \& federal institutions (D)
  \item
    women reviewers are more likely to recommend acceptance for men from
    R1 \& industry instutitions (C)
  \item
    men are more likely to recommend acceptance for women from R2
    institutions \& favor men from R1 \& medical institutes (D)
  \end{itemize}
\end{itemize}

\emph{Factors affecting competency}

\includegraphics{Figure_8.png}

\textbf{Figure 8. Comparison of time to final decision and impact by
gender.} The number days (A) between when a manuscript is initally
submitted then finally published and (B) that a manuscript spends in the
ASM peer review system. How the impact of papers published by men (blue)
versus women (orange) vary according to (C) cites and (D) total reads.
Citation data includes articles published between 36 and 48 months prior
to August 2018. Total reads includes both HTML and PDF online views for
articles published between 12 and 24 months prior to August 2018. Impact
data are divided by the number of months published.

\begin{itemize}
\tightlist
\item
  In addition to rejectance/acceptance rates, other disparate outcomes
  may occur during the peer review process
\item
  Time in Peer Reivew (time\_to\_publication.R)

  \begin{itemize}
  \tightlist
  \item
    Papers published by women take slightly longer (from submission to
    ready for publication) than men at some journals (mSphere, mBio,
    mSystems, CVI, JB, JCM, AEM) despite spending similar amounts of
    time at ASM journals (A, B)
  \item
    but do not require a greater number of revisions to be accepted than
    men (Supplementary\_C)
  \item
    Papers rejected following review that were submitted by women do not
    generally take longer (in days) to be rejected (except at mSystems)
    or have more revisions (Supplementary\_D \& \_E)
  \item
    Decisions are returned to men \& women similarly (time\_at\_journ.R)
  \end{itemize}
\item
  Peer review is not a linear process, it is cyclical and builds upon
  its self, may also be disparities in impact (impact\_analysis.R)

  \begin{itemize}
  \tightlist
  \item
    Women tend to receive lower cites per month published than men
  \item
    HTML/PDF/Abstract views are equivalent
  \item
    women are more likely to be published in lower JIF journals
  \item
    No disparity in gendered sumissions to high (mBio) vs lower
    (mSphere) journals (bias\_by\_jif.R)
  \end{itemize}
\end{itemize}

\subsection{Discussion}\label{discussion}

\begin{itemize}
\tightlist
\item
  Summarize results

  \begin{itemize}
  \tightlist
  \item
    Gender

    \begin{itemize}
    \tightlist
    \item
      Women/men/unclear are X percent of authors, reviewers, editors
    \item
      Women/men/unclear are more likely to be repeat authors/submitters
    \item
      M are more likely than W to be suggested as reviewers and are/are
      not used as reviewers at the same proportion
    \item
      M/F editors are more likely to handle multiple manuscripts --
      depends on journal
    \item
      Women are not retained to the same extent as men/unclear
    \item
      These observations do/do not correlate with gender of EiC
    \item
      gap between men \& women peformance, rejection rates
    \item
      women more likely to be editorially rejected or given rejection
      after review
    \end{itemize}
  \item
    Compare to global and ASM membership stats
  \item
    Globally - microbiology researchers are 60:40, M:F - Elsevier
  \item
    ASM membership - 38.37 (sept 2018)
  \end{itemize}
\end{itemize}

The under representation of women as corresponding authors in
publication at ASM journals has negative consequences for their careers
and microbiology. Buckley et al, suggest that being selected as a
reviewer increases visibility of a researcher, which has a direct \&
significant impact on salary. Therefore, the underrepresentation of
women as reviewers hampers their career progression and even their
desire to progress since reviewing also signals adoption of the
researcher into the scientific community (Buckley et al, 2014). This is
supported by Lerback and Hanson who noted that ``It {[}reviewing{]}
provides positive feedback that a scholar is respected and participating
in their field and fosters self-confidence, all of which lead to
increased retention of women.'' (Lerback \& Hanson, 2017) Retention of
women in science is important to the progress of microbiology as a field
since less diversity in researchers limits the diversity of
perspectives, approaches, and thus stunts the search for knowledge. In
addition to boosting productivity and knowledge, more diverse and
equitable organizations are more inclusive and supportive for all
members (Potvin, 2018). It is thus a moral and scientific imperity for
scientific societies and journals, such as ASM, to improve its own
diversity, equity and inclusion efforts. The remainder of this
manuscript will focus on actions that can be taken at multiple levels of
the peer review system to support these efforts.

Certain attributes of biological scientific societies correlate with
increased gender representation at leadership levels (Potvin, 2018).
Using the scientific society ``health checklist'' developed from these
observations, we propose the following suggestions to improve
representation at society journals. First, the development of a visible
mission, vision, or other commitment to equity and inclusion that
includes a non-discrimination clause regarding decisions made by editors
and editors-in-chief. This non-discrimination clause would be backed by
a specific protocol for the reporting of, and responding to, instances
of discrimination and harassment. In the long term, society journals
should begin collecting additional data about authors and gatekeepers
(e.g., race, ethnicity, sexual orientation, gender identity, and
disabilities). Such author data should not be readily available to
journal gatekeepers, but instead kept in a disagreggated manner that
allows the public presentation to track success of inclusive measures
and maintain accountability. Society journals can also impliment
mechanisms to explicity provide support for women and other minority
groups, e.g, by providing APC waivers, reduced copyediting services,
reward inclusive behavior by gatekeepers, encourage women to take up
leadership positions and provide gender-neutral, non-exclusive social
activities.

A common debate when filling leadership positions is whether they should
be representational of the field or aspirational. For instance, since
X\% of corresponding authors to ASM journals are women, X\% of
gatekeepers of a representational leadership would be women. Conversely,
50\% of gatekeepers would be women if the goal were an aspirational
leadership. We argue that whether a goal should be representational or
aspirational depends on the workload and visibility of the position(s).
Since high visibility positions (e.g., editor, EIC) are filled by a
smaller number of individuals that are responsible for recruiting more
individuals into leadership, filling these positions should be done
aspirationally. This allows expansion of the potential reviewer network
and thus recruitment into those positions. These lower visibility
positions (e.g., reviewers) require a greater number of individuals and
should thus be representational of the field to avoid overburdening the
minority population. Outside of leadership appointment, all parties,
journals, gatekeepers, and authors, can help advance women (and other
minority groups) within the peer review system. For instance, authors
can suggest more women as reviewers using ``Diversify'' resources (e.g.,
DiversifyMicrobiology), while reviewers can agree to review for women
editors more often. Editors can rely more on manuscript reference lists
and data base searches than personal knowledge (Fox et al, 2016), and
journals can improve the interactivty and functionality of the peer
review selection software.

Addressing bias during peer review process is a more difficult
challenge, since it is partially the result of accumulated disadvangates
and microaggressions (the actions resulting from implict biases).
Implict bias training for gatekeepers is a start, as might be
double-blinded peer review, a common practice in social sciences. To
support efforts of making peer review more transparent, the review
process could be unblinded following the editor's final decision on a
manuscript. However, these solutions are only bandaids on a deeply
infected wound since both focus on the superficial issue of individuals
instead of the underlying structure of the system that has selected for
the bias at hand.

Reconsidering journal scope and the overall attitude toward
replicatitive and negative results might help address structural
barriers to representation of women in peer review. Significant funds
and staff are required to be competitive in highly active fields (e.g.,
\emph{Clostridium difficile}, HIV), but women are often at a
disadvantage for these resources. As a result, corresponding authors
that are women may be spending their resources at the lesser competitive
fringes of research fields. This has the disadvange of making them seem
``less competent'' to those at the established center of the field. The
decrease in percieved researcher competency and research validity
increase the difficulty to obtain funding and publish in more
traditional journals. Expanding journal scope could provide a home for
these innovative research fields, bolster the field through
reproduciblity, and improve the competentcy demonstration of these
researchers.

Few papers have found disparities between rejection rates of men and
women and to our knowledge, this is the first paper to collectively
examine this issue with either submissions data from 10+ journals or on
the field of microbiology. Critics might argue that the effect size is
too small to really matter or that there are too many unaccounted for
factors to draw conclusions. We acknowledge that these are limitations
of our study along with a limited journal dataset, an absence of
reviewer comments for sentiment analysis, and that many ASM journals
have a narrow focus while the broad scope journals are relatively new.
All of these factors prevent us from generalizing our results across
microbiology as a field. However, the consistency of the trends to
benefit men corresponding authors over women, across all journals
included and literature to-date confirms that this study is highly
relevant for the ASM as a society and offers opportunities to address
both gendered representation in microbiology and systemic barriers to
peer review at our journals.

\subsection{Data and Methods}\label{data-and-methods}

\emph{Data}

All manuscripts handled by ASM journals (e.g., \emph{mBio},
\emph{Journal of Virology}) that received an editorial decision between
January 1st, 2012 and August 31st, 2018 were supplied as XML files by
ASM's publishing platform, eJP. Data were extracted from the XML
documents provided using R statistical software (version 3.4.4) and the
\texttt{XML} package (R citation). Data manipulation was handled using
the \texttt{tidyverse}, \texttt{lubridate}, and \texttt{xml2} packages
for R. Variables of interest included: the manuscript number assigned to
each submission, manuscript type (e.g., full length research, erratum,
editorial), category (e.g., microbial ecology), related (previously
submitted) manuscripts, versions submitted, dates (e.g., submission,
decision), author data (e.g., first, last, and corresponding authorship,
total number of authors), reviewer data (e.g., reviewer score,
recommendation, editor decision), and person data (names, institutions,
country) of the editors, authors, and reviewers.

For this analysis, only original, research-based manuscripts were
included, e.g., long- and short-form research articles, New-Data
Letters, Observations, Opinion/Hypothesis articles, and Fast-Track
Communications.

It is common practice at ASM journals for manuscripts whose reviewers
recommend extensive experimental revisions be given a decision of
``reject with resubmission encouraged''. If resubmitted, the authors are
asked to note the previous (related) manuscript and the resubmission is
assigned a new manuscript number. Multiple related manuscripts were
tracked together by generating a unique grouped manuscript number based
on the recorded related manuscript numbers. This grouped manuscript
number served multiple purposes including: tracking a single manuscript
through multiple rejections or transfers between ASM journals and to
avoid duplicate counts of the same authors for the same manuscript.

Data were visualized using the \texttt{ggplot}, \texttt{scales},
\texttt{RColorBrewer}, and \texttt{ggalluvial} packages for R.

\emph{Bias analysis and presentation}

\emph{Logistic regression and clustering}

\emph{Gender prediction and assignment}

The gender assignment API genderize.io was used to predict an
individual's gender based on their given names, and country where
possible. The genderize.io platform uses data gathered from social media
to predict gender based on given names with the option to include an
associated language or country to enhance the odds of successful
prediction. Since all manuscripts are submitted in English, precluding
language association for names with special characters, names were
standardized to ASCII coding (e.g., ``José'' to ``Jose''). We next
matched each individuals country against the list of X country names
accepted by genderize.io. Using the \texttt{GenderGuesser} package for
R, all unique given names associated with an accepted country were
submitted to the genderize.io API and any names returned without a
predictive assignment of either male or female were resubmitted without
an associated country. All predictive assignments of either male or
female are returned with a probability match of 0.50 or greater. The
predicted genders of all given names (with and without an associated
country) whose probabilities were greater or equal to our arbitrary
success cut off of 0.65 were used to assign predicted gender to the
individuals in our dataset. Predicted genders were assigned to
individuals in the following order: first names and country, first
names, middle names and country, middle names (Supplemental Figure 1).
The presenting gender (man/woman) of editors and senior editors in our
dataset was hand validated using Google where possible.

We recognize that biological sex (male/female) is not always equivalent
to the gender that an individual presents as (man/woman), which is also
distinct from the gender(s) that an individual may self-identify as. For
the purposes of this manuscript, we choose to focus on the presenting
gender (man/woman/unclear) based on their first names and/or appearance
(for editors). In the interest of transparency, we include those
individuals whose names don't allow a high degree of confidence for
gender assignment in the ``unclear'' category of our analysis.

\emph{Validation of gender prediction}

We first validated the algorithm using a set of 3265 names whose gender
had been hand-coded based on appearance and were generously provided to
us by \_\_\_ (preprint cite). The names were supplied to the genderize
algorithm both with and without the accompanying country data. The data
returned include the name, predicted gender (male, female, na), the
probability of correct gender assignment (ranging from 0.5 to 1.0), and
the number of instances the name and gender were associated together (1
or greater). The genderize algorithm returned gender predictions for
2899 when first names were given and 2167 when country data was also
supplied (732 names were associated with countries unsupported by
genderize).

Sensitivity and specificity, are measurements of the algorithm's
tendency to return correct answers instead of false positives (e.g., a
man incorrectly gendered as a woman) or false negatives (e.g., a woman
incorrectly gendered as a man). The closer these values are to 1, the
smaller the chance that the algorithm will return the correlating false
response. Accuracy is a composite measure of the algorithm's ability to
differentiate the genders correctly. These measurements were calculated
from the datasets (with and without country data supplied) at three
different probability threshold cutoffs: the default genderize (0.5), a
probability threshold of 0.85 (0.85), and a modified probability of
0.85, which factors in the number of instances returned
(pmod0.85)(citations).

At the 0.5 threshold, the dataset returned a sensitivity of 0.8943 and
specificity of 0.9339 for an accuracy of 0.911, compared to a marginally
higher accuracy of 0.9146 for the dataset where country data were
included (Supplemental Table 1). Generally speaking, the accuracy
increases as the threshold increases along with slight trade offs
between sensitivity and specificity. For the purposes of our analysis,
we opted to use the pmod0.85 threshold moving forward (Supplemental
Table 1, in bold).

To understand the extent of geographic bias in our gender assignment
against regions and languages with genderless naming conventions, or
that lack social media for incorporation into the genderize algorithm,
we compared the number of names predicted without associated country
data to when country data was also supplied. In our test dataset, the
top five countries associated with names were United States, Germany,
United Kingdom, France, and China and the countries with the highest
proportion of un-predicted genders when country data were supplied are
Cambodia, Iceland, Indonesia, Ireland, and Mexico, where the maximum
number of names supplied ranged from 1 to 15. To determine the impact of
each country towards the overall percentage of names whose genders were
not predicted (27.14\%), we found the difference between the percent of
names unpredicted for each country and the overall percentage,
multiplied by the proportion of observations from that country to the
total observations and finally divided by the overall percentage of
unpredicted names (Supplemental Figure 2). The top five countries with
the greatest impact on unpredicted names, and thus the countries
receiving the most negative bias from genderize were Canada, China,
Ireland, Belgium, and Sweden (Supplemental Figure 3). These data suggest
that there is likely some bias against countries with gender-neutral
naming conventions (China), and indicates the stringency with which the
algorithm applies gender to names that are accompanied by country data.
For instance, strongly gendered names such as Peter and Pedro were not
assigned gender when associated with Canada.

We next applied the genderize algorithm at the pmod0.85 threshold to our
journals dataset and tested its validity on a small portion. All first
names collected from our dataset were submitted to genderize both with
and without country data. Only those predictions whose pmod were
equivalent or greater than 0.85 were carried to the next step. The
predicted genders were assigned to individuals in the following order:
first names and country, first names, middle names and country, middle
names. Given the relatively small number of editors and senior editors
in our dataset, the presenting gender (man/woman) of editors and senior
editors in our dataset was hand-validated using Google where possible.
Of the 1072 editor names, 938 were predicted by genderize for an
accuracy of 0.9989339, thus increasing our confidence in the gender
predictions where made.

In our full dataset, the five countries with the most individuals were
United States, China, Japan, France, and Germany and the countries with
the highest proportion of un-predicted genders were Burundi, Chad,
Kingman Reef, Korea (North), Democratic People's Republic of, and
Maldives, where the maximum number of names supplied ranged from 1 to 4.
Proportionally, fewer names in our full dataset were assigned gender
than in our validation dataset (40.01\% unpredicted versus 27.14\%
unpredicted, respectively). Since adjusting the workflow to predict the
gender of names both with and without country data, the countries
receiving the most negative bias from genderize were China, Japan,
Korea, Republic of, India, Taiwan, Province of China (Supplemental
Figure 4). These data indicate what we previously predicted, that the
genderize algorithm has bias against countries with gender-neutral
naming conventions.

\subsection{References}\label{references}


\end{document}
